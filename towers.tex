\RequirePackage[l2tabu, orthodox]{nag}
\documentclass[11pt,headsepline=true,toc=flat]{scrartcl}

\usepackage{xspace, xifthen, enumitem}

\usepackage{amssymb, amsmath, amsthm, thmtools, nth}

\usepackage{graphicx}

\usepackage{etoolbox}

\usepackage{tikz}
\usetikzlibrary{arrows,decorations.markings,chains,calc,matrix}
\usepackage{tikz-cd}
\tikzset{>=cm to}

\usepackage[lining]{libertine}
\usepackage[T1]{fontenc}
\usepackage{textcomp}
\usepackage[varqu,varl]{inconsolata}
\usepackage[italic, basic, eulergreek, defaultmathsizes]{mathastext}
\usepackage{bm}
\usepackage{mathtools}
\mathtoolsset{mathic}

\usepackage[cal=boondoxo]{mathalfa}
\usepackage{mathrsfs}
\newcommand{\mathcalls}[1]{{\textls*[-150]{\usefont{U}{BOONDOX-calo}{m}{n} #1}}}

\usepackage[colorlinks=true]{hyperref}
\usepackage[all]{hypcap}

\usepackage{csquotes}
\usepackage[english]{babel}
\usepackage[nodayofweek]{datetime}

\usepackage[protrusion=true]{microtype}

\usepackage[bibencoding=utf8,style=alphabetic,citestyle=alphabetic,backref=true,hyperref=true,firstinits=true,doi=true]{biblatex}
\addbibresource{all.bib}
\renewcommand*{\bibfont}{\normalfont\footnotesize}
\renewbibmacro{in:}{%
  \ifentrytype{article}{}{\printtext{\bibstring{in}\intitlepunct}}}
\renewrobustcmd*{\bibinitdelim}{\,}
\AtEveryBibitem{%
  \clearfield{pagetotal}%
}

\usepackage{pdfpages}

\declaretheoremstyle[spaceabove=\topsep,spacebelow=\topsep,headfont=\normalfont\scshape,notefont=\normalfont\mdseries,notebraces={(}{)},bodyfont=\normalfont,postheadspace=5pt plus 1pt minus 1pt]{scdef}
\declaretheoremstyle[spaceabove=\topsep,spacebelow=\topsep,headfont=\normalfont\scshape,notefont=\normalfont\mdseries,notebraces={(}{)},bodyfont=\itshape,postheadspace=5pt plus 1pt minus 1pt]{scthm}

\declaretheorem[style=scdef,numberwithin=section,   name=Definition,refname={definition,definitions},Refname={Definition,Definitions}]{definition}
\declaretheorem[style=scdef,sharenumber=definition, name=Remark,refname={remark,remarks},Refname={Remark,Remarks}]{remark}
\declaretheorem[style=scdef,sharenumber=definition, name=Example,refname={example,examples},Refname={Example,Examples}]{example}

\declaretheorem[style=scthm,sharenumber=definition, name=Theorem,refname={theorem,theorems},Refname={Theorem,Theorems}]{theorem}
\declaretheorem[style=scthm,sharenumber=definition, name=Lemma,refname={lemma,lemmas},Refname={Lemma,Lemmas}]{lemma}
\declaretheorem[style=scthm,sharenumber=definition, name=Corollary,refname={corollary,corollaries},Refname={Corollary,Corollaries}]{corollary}
\declaretheorem[style=scthm,sharenumber=definition, name=Proposition,refname={proposition,propositions},Refname={Proposition,Propositions}]{proposition}

\undef\Re
\undef\Im

\newcommand*{\normal}{\lhd}
\newcommand*{\isom}{\cong}
\newcommand*{\homot}{\sim}
\newcommand*{\wequiv}{\simeq}

\makeatletter
\let\@oldsubset=\subset
\def\@subsethelper#1#2{\mathrel{\raisebox{.5pt}{$#1\@oldsubset$}}\xspace}
\DeclareRobustCommand*{\subset}{\mathpalette\@subsethelper\relax}

\let\@oldotimes=\otimes
\def\@otimeshelper#1#2{\mathrel{\raisebox{.5pt}{$#1\@oldotimes$}}\xspace}
\DeclareRobustCommand*{\otimes}{\mathpalette\@otimeshelper\relax}
\makeatother

\tikzset{/tikz/commutative diagrams/arrows={thin}}

\newcommand*{\ca}[1]{\ensuremath{\mathscr{#1}}\xspace}
\renewcommand*{\cal}[1]{\ensuremath{\mathcal{#1}}\xspace}
\newcommand*{\f}[1]{\ensuremath{\mathfrak{#1}}\xspace}

\newcommand{\cl}[2][0]{{}\mkern#1mu\overline{\mkern-#1mu#2}}
\newcommand*{\Int}[1]{\ensuremath{#1^\circ}\xspace}

\newcommand*{\ie}{i.\,e.}
\newcommand*{\eg}{e.\,g.}
\newcommand*{\Ie}{I.\,e.}
\newcommand*{\Eg}{E.\,g.}

\undef\lrcorner
\newcommand{\lrcorner}{\mathord{\vrule height 0.1ex depth 0pt width 1ex\vrule height 1.3ex depth 0pt width
0.1ex}}

\def\<#1>{\left\langle #1 \right\rangle}

\undef\AA
\undef\SS
\renewcommand*{\do}[1]{\expandafter\def\csname#1#1\endcsname{\ensuremath{\mathbb{#1}}\xspace}}
\docsvlist{A,B,C,D,E,F,G,H,I,J,K,L,M,N,O,P,Q,R,S,T,U,V,W,X,Y,Z}

\setlist[enumerate]{label={\normalfont \rmfamily(\roman*)}, nosep}
\setlist[itemize]{nosep}

\overfullrule=1mm

\usepackage{todonotes}
\newcommand{\aravind}[1]{\todo[color=red!40]{#1}} %notes by Aravind
\newcommand{\marc}[1]{\todo[color=blue!40]{#1}} %notes by Marc
\newcommand{\matthias}[1]{\todo[color=cyan!40]{#1}} %notes by Matthias
\newcommand{\brad}[1]{\todo[color=magenta!40]{#1}} %notes by Brad
\newcommand{\viktor}[1]{\todo[color=yellow!40]{#1}} %notes by Viktor

\undef\lim
\DeclareMathOperator*{\lim}{\textnormal{colim}}
\DeclareMathOperator*{\colim}{\textnormal{colim}}
\DeclareMathOperator*{\hocolim}{\textnormal{hocolim}}
\DeclareMathOperator*{\holim}{\textnormal{holim}}

\DeclareMathOperator{\sk}{\textnormal{sk}}
\DeclareMathOperator{\cosk}{\textnormal{cosk}}

\undef\hom
\DeclareMathOperator{\hom}{\textnormal{Hom}}
\DeclareMathOperator{\Hom}{\textnormal{\bfseries Hom}}
\DeclareMathOperator{\RHom}{\mathbb{R}\textnormal{\bfseries Hom}}

\DeclareMathOperator{\fib}{\textnormal{fib}}
\DeclareMathOperator{\cof}{\textnormal{cof}}

\newcommand{\trunc}[1]{\tau_{#1}}



\bibliography{all}
\begin{document}

\title{Postnikov Towers in Combinatorial Simplicial Model Categories}
\date{}
\maketitle

{\footnotesize
\tableofcontents
}

\section{Truncated and Connected Simplicial Sets}

\begin{definition}\label{defn:truncated_simp}
  A simplicial set \(X\) is called \emph{\(k\)--truncated} for
  \(k\geq-1\) if \(\pi_{n}(X,x) = *\) for every \(x\in X_{0}\) and
  every \(n>k\). A morphism \(f\colon Y\to X\) of simplicial sets is
  called \emph{\(k\)--truncated} if every homotopy fiber of \(f\)
  is \(k\)--truncated.
\end{definition}

By convention we call a simplicial set \((-2)\)--truncated
if and only if it is contractible; a map of simplicial sets is
\((-2)\)--truncated if it is a weak equivalence.

\begin{lemma}\label{lem:truncated-diagonal}
  A simplicial set \(X\) is \(k\)--truncated if and only if the
  diagonal \(\Delta\colon X\to X\times X\) is \((k-1)\)--truncated.
\end{lemma}
\begin{proof}
  Let \(X\) be any connected simplicial set and let \((a,b)\in X\times
  X\). Let \(P\) be the homotopy fiber of \(\Delta\) at \((a,b)\).
  Any path from \(a\) to \(b\) in \(X\) induces a homotopy equivalence
  \(\Omega_{a}X \simeq P \simeq \Omega_{b} X\), hence isomorphisms
  \(\pi_{n+1}(X)\cong \pi_{n}(P)\). This proves the claim for
  connected simplicial sets. For the general case, write \(X =
  \bigsqcup_{i\in I} X_{i}\) as a disjoint union of its connected
  components and observe that \(X\times X = \bigsqcup_{i,j\in I}
  X_{i}\times X_{j}\). Then the homotopy fiber of the diagonal over a
  point \((a,b)\in X_{i}\times X_{j}\) is either empty (if \(i\neq
  j\)) or homotopy equivalent to \(\Omega_{a}X_{i}\) (if \(i =
  j\)). Since \(X\) is \(k\)--truncated if and only if all its
  connected components are \(k\)--truncated, this reduces the problem
  to the case of connected simplicial sets, which we have already
  dealt with.

\end{proof}

\begin{lemma}\label{lem:relative-truncated-diagonal}
  A map \(f\colon Y\to X\) of simplicial sets is \(k\)--truncated if
  and only if the diagonal \(\Delta_{f}\colon Y\to Y\times^{h}_{X} Y\)
  is \((k-1)\)--truncated.
\end{lemma}
\begin{proof}
  We can assume that \(f\) is a fibration between fibrant simplicial
  sets. If \(x\in X\) then the homotopy fiber of \(f\) is just the
  ordinary fiber product \(Y_{x}\coloneqq \{x\}\times_{X} Y\). Observe
  that \(\Delta_{f}\) is weakly equivalent to the fibration
  \(Y^{\Delta^{1}}\to Y\times_{X} Y\) and \(Y\times_{X}^{h} Y \simeq
  Y\times_{X} Y\). Given any \((y, y')\in Y\times_{X} Y\), let \(x =
  f(y) = f(y')\). Then there is a commutative diagram
  \[
  \begin{tikzcd}
    \Omega_{y} Y_{x} \arrow[r] \arrow[d] & Y_{x}^{\Delta^{1}} \arrow[r] \arrow[d] & Y^{\Delta^{1}} \arrow[d] \\
    * \arrow[r, "{(y,y')}"'] & Y_{x}\times Y_{x} \arrow[r] & Y\times_{X} Y
  \end{tikzcd}
  \]
  in which both small squares are cartesian. It follows that the
  homotopy fiber of \(\Delta_{f}\) at \((y,y')\) is weakly equivalent
  to the homotopy fiber of \(\Delta\colon Y_{x}\to Y_{x}\times
  Y_{x}\). Hence, all homotopy fibers of \(\Delta_{f}\) are
  \((k-1)\)--truncated if and only if all homotopy fibers of all
  diagonals \(Y_{x}\to Y_{x}\times Y_{x}\) are. By
  \autoref{lem:truncated-diagonal} this is the case if and only if all
  \(Y_{x}\) are \(k\)--truncated, i.\,e.~if and only if the map \(Y\to
  X\) is \(k\)--truncated.
\end{proof}

\begin{corollary}\label{cor:truncatedness-makes-sense}
  A simplicial set \(X\) is \(k\)--truncated if and only if for every
  simplicial set \(Z\) the derived mapping space \(\RHom(Z, X)\) is
  \(k\)--truncated. A map \(f\colon X\to Y\) is \(k\)--truncated if
  and only if for every \(Z\) the map \(f_{*}\colon \RHom(Z,
  X)\to\RHom(Z, Y)\) is \(k\)--truncated.
\end{corollary}
\begin{proof}
  The proof proceeds by induction on \(k\). By
  \autoref{lem:relative-truncated-diagonal} a map \(f\colon Y\to X\)
  is \(k\)--truncated if and only if \(\Delta_{f}\colon Y\to
  Y\times_{X}^{h} Y\) is \((k-1)\)--truncated. By induction this is
  the case if and only if \(\RHom(Z,Y)\to \RHom(Z, Y\times_{X}^{h}
  Y)\) is \((k-1)\)--truncated. Observe that
  \[
  \begin{tikzcd}
    \RHom(Z,Y) \arrow[r] \arrow[dr] & \RHom(Z, Y\times^{h}_{X} Y) \arrow[d, "\simeq"] \\
    {} & \RHom(Z, Y)\times_{\RHom(Z, X)}^{h} \RHom(Z,Y)
  \end{tikzcd}
  \]
  commutes. Hence, \(\Delta_{f_{*}}\) is \((k-1)\)--truncated as well
  and again by \autoref{lem:relative-truncated-diagonal} this is
  equivalent to \(f_{*}\colon \RHom(Z, Y)\to\RHom(Z, X)\) being
  \(k\)--truncated.
\end{proof}

We will now constructe explicit \emph{truncation functors} for the
category \(\sset\) of simplicial sets. By this we mean functors
\(\tau_k\colon \sset\to\sset\) such that \(\tau_kX\) is
\(k\)--truncated for every \(X\in\sset\) and universal in some
suitable sense.

First, the simplex category \(\simplex\), when regarded as the
category of nonzero finite ordinals, has a natural filtration by full
subcategories
\[
* = \simplex_{\leq 0} \subset \simplex_{\leq 1} \subset\cdots\subset\simplex
\]
where \(\simplex_k\) is the category of nonzero finite ordinals less than
or equal to \(k\). This induces a tower of functors
\[
\sset \to \cdots \to \sset_{\leq 1} \to \sset_{\leq 0} = \set
\]
and we write \(\iota_k^*\colon \sset \to \sset_{\leq k}\) for the
restriction functors.

By general abstract nonsense the functors \(\iota_k^*\) extend to
adjunctions
\[
(\iota_k)_! \dashv \iota_k^* \dashv (\iota_k)_*,
\]
see e.\,g.~\cite[Theorem~4, p.~59]{sheavesgeometrylogic} or
\cite[{Exposé}~i, Proposition~5.1]{SGA4-1}. Composing these, we get an
adjunction
\[
\sk_k = (\iota_k)_{!}\iota_k^{*} \dashv (\iota_k)_{*}\iota_k^{*} = \cosk_k.
\]

\begin{proposition}\label{prop:cosk-is-truncated}
  If \(X\in\sset_{\leq k}\), then \((\iota_k)_{*}X \in \sset\) is
  \(k\)--truncated. In particular, \(\cosk_k(Y)\) is \(k\)--truncated
  for all simplicial sets \(Y\).
\end{proposition}

\viktor{Prove that \(\cosk_{n}\) are truncation functors. Prove
  classical homotopy group statements.}

\section{Truncated and Connected Morphisms}
Most of the material in this section is due to \cite{mr2522659} and
\cite{rezkhomotopytoposes}. For the benefit of the reader\footnote{and
  for our own benefit} we try to elaborate on results whose proofs are
only sketched in these references.

\begin{definition}\label{defn:truncated_object}
  Let \(\mathbf{M}\) be a combinatorial simplicial model category. An
  object \(X\in\mathbf{M}\) is called \emph{\(k\)--truncated} (\(k\geq
  -1\)) if for all objects \(Y\in\mathbf{M}\) the derived mapping
  space \(\RHom(Y,X)\) is a \(k\)--truncated simplicial set. A
  morphism \(f\colon Y\to X\) in \(\mathbf{M}\) is called
  \emph{\(k\)--truncated} if for all objects \(Z\in\mathbf{M}\) the
  induced morphism \(f_{*}\colon \RHom(Z,Y)\to\RHom(Z,X)\) is
  \(k\)--truncated.
\end{definition}

\begin{remark}
  An object \(X\in\mathbf M\) is \(k\)--truncated if and only if the
  morphism \(X\to *\) is \(k\)--truncated.
\end{remark}

As a matter of convention, an object \(X\) will be called
\((-2)\)--truncated if it is contractible and a morphism \(f\colon
Y\to X\) will be called \((-2)\)--truncated if it is a weak
equivalence.

\begin{corollary}\label{cor:general-diagonal-truncated}
  In any combinatorial simplicial model category, a morphism \(f\colon
  Y\to X\) is \(k\)--truncated if and only if the diagonal
  \(\Delta_{f}\colon Y\to Y\times^{h}_{X} Y\) is
  \((k-1)\)--truncated.
\end{corollary}
\begin{proof}
  This follows immediately from \autoref{defn:truncated_object} and
  \autoref{lem:relative-truncated-diagonal}.
\end{proof}

\begin{definition}
  Let \(\mathbf{M}\) be a combinatorial simplicial model
  category. Then \(\mathbf{M}|_{{\leq}k}\) will denote the full
  subcategory on \(k\)--truncated objects of \(\mathbf{M}\).
\end{definition}

Recall that a \emph{relative category} is a category \(\mathbf{M}\)
together with a subcategory \(W\subseteq \mathbf{M}\) containing all
objects of \(\mathbf{M}\). For instance, every model category with the
subcategory of weak equivalences is a relative category. A
\emph{relative functor} \(F\colon (\mathbf{M},W) \to (\mathbf{N},W')\)
is a functor \(\mathbf{M}\to\mathbf{N}\) such that \(F(W)\subset
W'\). A relative functor between model categories is sometimes also
called a \emph{homotopical functor}.

\begin{remark}\label{rem:holim-homotopical}
  Let \(I\) be some category and let \(\mathbf{M}^{I}\) be the
  category of \(I\)--shaped diagrams in \(\mathbf{M}\). In
  \(\mathbf{M}^{I}\) we take the weak equivalences to be those natural
  transformations that induce weak equivalences in \(\mathbf{M}\) on
  every object \(i\in I\). The Bousfield--Kan construction of the
  homotopy limit yields a functor
\[
\holim\colon \mathbf{M}^I \to \mathbf{M}
\]
  In fact, this functor is homotopical by the weak homotopy invariance
  of the homotopy limit.
\end{remark}

\begin{proposition}\label{prop:truncated-objects-exact-functors}
  Let \(F\colon \mathbf{M}\to\mathbf{N}\) be a homotopical functor
  between combinatorial simplicial model categories that preserves
  homotopy pullbacks; that is, the image under \(F\) of any homotopy
  pullback square in \(\mathbf M\) is a homotopy pullback square in
  \(\mathbf N\).

  Then \(F\) preserves \(k\)--truncated morphisms for any \(k\) and
  consequently restricts to a functor
  \(F\colon \mathbf M|_{\leq k}\to\mathbf N|_{\leq k}\).
\end{proposition}
\begin{proof}
  Since \(F\) is homotopical, by definition it preserves
  \((-2)\)--truncated morphisms. Let \(k\geq -2\) and suppose by
  induction that \(F\) preserves \(n\)--truncated morphisms for
  \(n<k\). Let \(f\colon Y\to X\) be \(k\)--truncated in \(\mathbf
  M\). By \autoref{cor:general-diagonal-truncated} the diagonal
  \(\Delta_{f}\colon Y\to Y\times^{h}_{X} Y\) is \((k-1)\)--truncated
  and by induction \(F(\Delta_{f})\) is \((k-1)\)--truncated. Since
  \(F\) preserves homotopy pullbacks \(F(\Delta_{f})\) is weakly
  equivalent to \(\Delta_{F(f)}\). Again by
  \autoref{cor:general-diagonal-truncated} we conclude that \(F(f)\)
  is \(k\)--truncated.
\end{proof}

\begin{corollary}\label{cor:truncated-objects-limits}
  The full subcategory \(\mathbf{M}|_{{\leq} k}\) is stable under
  homotopy limits.
\end{corollary}
\begin{proof}
  Forming homotopy limits is an example of a homotopical functor, see~\autoref{rem:holim-homotopical}, which preserves homotopy pullbacks.
\end{proof}

\begin{theorem}\label{thm:truncated-model-structure-exists}
  Let \(\mathbf{M}\) be a combinatorial simplicial model category and
  \(k\geq -1\). There is a combinatorial simplicial model structure on
  \(\mathbf{M}\) whose fibrant objects are exactly the
  \(k\)--truncated fibrant objects of \(\mathbf{M}\). The resulting
  model category will be denoted by \(\mathbf{M}_{\leq k}\).
\end{theorem}
\begin{proof}
  By \cite[Prop.~4.7]{mr1870516} there is a set \(A\) of objects of
  \(\mathbf{M}\) such that every \(X\in\mathbf{M}\) is canonically
  weakly equivalent to a homotopy colimit of objects in \(A\). Define
  a set of morphisms
  \[
  T \coloneqq \{S^{n}\times a \to D^{n+1}\times a : a\in A, n > k \}
  \]
  in \(\mathbf{M}\). Recall that an object \(X\in\mathbf{M}\) is
  called \(T\)--local if for every \(f\colon S^{n}\times a\to
  D^{n+1}\times a\) in \(T\) the induced map
  \[
  f^{*}\colon \RHom(D^{n+1}\times a, X)\to\RHom(S^{n}\times a, X)
  \]
  is a weak equivalence of simplicial sets. We claim that the
  \(T\)--local objects of \(\mathbf{M}\) are precisely the
  \(k\)--truncated objects. To see this, assume first that \(X\) is
  \(T\)--local and let \(Y\in\mathbf{M}\) be any object and let
  \(\{a_{i}\}\colon I\to\mathbf{M}\) be a diagram of objects
  \(a_{i}\in A\) such that \(Y\simeq \hocolim_{i\in I} a_{i}\). Then,
  for \(n>k\), we have
  \begin{align*}
    \RHom(Y, X^{S^{n}}) &\simeq \RHom(\hocolim_{i\in I}a_{i}, X^{S^{n}})\simeq \holim_{i\in I}\RHom(a_{i}, X^{S^{n}}) \simeq \\
                        &\simeq \holim_{i\in I} \RHom(S^{n}\times a_{i}, X) \simeq \holim_{i\in I}\RHom(D^{n+1}\times a_{i}, X)\simeq \\
                        &\simeq \holim_{i\in I}\RHom(a_{i}, X^{D^{n+1}})\simeq \RHom(Y, X^{D^{n+1}}).
  \end{align*}
  Consequently we find
  \begin{align*}
    \RHom(S^{n}, \RHom(Y, X)) & \simeq \RHom(Y, X^{S^{n}}) \simeq \RHom(Y, X^{D^{n+1}}) \simeq\\
                              &\simeq \RHom(D^{n+1}, \RHom(Y, X)).
  \end{align*}
  Hence, for any \(f\in\RHom(Y,X)_{0}\) we have
  \begin{align*}
    \pi_{n}(\RHom(Y, X), f) &\cong \pi_{0}\RHom_{*}(S^{n}, (\RHom(Y,X), f)) \cong \\
                            &\cong \pi_{0}(\RHom(S^{n}, \RHom(Y,X))\times^{h}_{\RHom(Y,X)}\{f\}) \cong \\
                            &\cong \pi_{0}(\RHom(D^{n+1}, \RHom(Y,X))\times^{h}_{\RHom(Y,X)}\{f\}) \cong \\
                            &\cong \pi_{0}\RHom_{*}(D^{n+1}, (\RHom(Y,X), f)) \cong \\
                            &\cong \pi_{0}\RHom_{*}(*, (\RHom(Y,X), f)) \cong \pi_{0}(*) = *.
  \end{align*}
  It follows that \(X\) is \(k\)--truncated.

  Conversely, assume \(X\) is \(k\)--truncated. This means that for
  any \(Y\in\mathbf{M}\) and any \(n>k\) the restriction map
  \[
  \RHom(D^{n+1}, \RHom(Y,X)) \to \RHom(S^{n}, \RHom(Y,X))
  \]
  is a weak equivalence. Hence, for every \(a\in A\) we find
  \begin{align*}
    \RHom(S^{n}\times a, X) &\simeq \RHom(S^{n}, \RHom(a, X))\simeq \\
    &\simeq \RHom(D^{n+1}, \RHom(a, X))\simeq \RHom(D^{n+1}\times a, X).
  \end{align*}
  It follows that \(X\) is \(T\)--local.

  Since \(T\) is a set of morphisms and \(\mathbf{M}\) is a
  combinatorial simplicial model category, the left Bousfield
  localization \(\mathbf{M}_{\leq k} \coloneqq L_{T}\mathbf{M}\) of
  \(\mathbf{M}\) at \(T\) exists and is itself a combinatorial
  simplicial model category, see
  \cite[Prop.~A.3.7.3]{mr2522659}. Furthermore, the same theorem also
  shows that the fibrant objects of \(\mathbf{M}_{\leq k}\) are
  precisely the fibrant \(T\)--local objects. But, by our previous
  calculations, these are exactly the \(k\)--truncated fibrant objects
  of \(\mathbf{M}\).
\end{proof}

\begin{definition}
  Let \(k\geq -2\) and \(f\colon X\to Y\) be a morphism in \(\mathbf
  M\). If for all \(Z\in\mathbf M|_{\leq k}\) the induced map
  \[f^{*}\colon \RHom_{\mathbf M}(Y, Z) \to \RHom_{\mathbf M}(X, Z)\]
  is a weak equivalence of simplicial sets, \(f\) will be a called
  \emph{\(k\)--connected} or a \emph{\(k\)--equivalence}. We write
  \(X\simeq_{\leq k} Y\).
\end{definition}

\begin{remark}
  Observe that every morphism is \((-2)\)--connected. A
  \((-1)\)--connected morphism is also called an \emph{effective
    epimorphism}.
\end{remark}

\begin{corollary}\label{cor:weak-equiv-level}
  The model category \(\mathbf M_{\leq k}\) is the left Bousfield
  localization of \(\mathbf M\) with respect to the collection of all
  \(k\)--equivalences.
\end{corollary}
\begin{proof}
  It will be enough to show that the weak equivalences in \(\mathbf
  M_{\leq k}\) are precisely the \(k\)--equivalences. To that end, let
  \(f\colon X\to Y\) be a morphism in \(\mathbf M\) and \(Z\in\mathbf
  M|_{\leq k}\). Let \(Z^{\fib}\) be a fibrant replacement of \(Z\) in
  \(\mathbf M\) and we can assume without loss of generality that
  \(X\) and \(Y\) are cofibrant. Because \(Z^{\fib}\) is
  \(k\)--truncated and fibrant in \(\mathbf M\), by
  \autoref{thm:truncated-model-structure-exists} it is also fibrant in
  \(\mathbf M_{\leq k}\). We obtain a commutative diagram
  \[
  \begin{tikzcd}
    \RHom_{\mathbf M}(Y, Z) \arrow[r, "f^{*}"] \arrow[d] & \RHom_{\mathbf M}(X, Z) \arrow[d] \\
    \Hom_{\mathbf M}(Y, Z^{\fib}) \arrow[r, "f^{*}"] & \Hom_{\mathbf M}(X, Z^{\fib}) \\
    \RHom_{\mathbf M_{\leq k}}(Y, Z) \arrow[r, "f^{*}"] \arrow[u] & \RHom_{\mathbf M_{\leq k}}(X, Z). \arrow[u]
  \end{tikzcd}
  \]
  in which all vertical maps are weak equivalences of simplicial
  sets. Hence, \(f\) is a weak equivalence in \(\mathbf M_{\leq k}\)
  if and only if for all \(Z\) the bottom map is a weak equivalence if
  and only if for all \(Z\) the top map is a weak equivalence if and
  only if \(f\) is a \(k\)--equivalence.
\end{proof}

\begin{remark}
  The model category \(\mathbf M_{\leq 0}\) is discrete in the
  following sense. Consider the full simplicial subcategory \((\mathbf
  M_{\leq 0})^{cf}\subset \mathbf M_{\leq 0}\) on the
  cofibrant--fibrant objects. Then a morphism \(f\colon X\to Y\) in
  \((\mathbf M_{\leq 0})^{cf}\) is a weak equivalence if and only if
  it is an isomorphism in \((\mathbf M_{\leq 0})^{cf}\). This follows
  from the fact that \(\pi_{0}(S) \isom S_{0}\) for \(0\)--truncated
  simplicial sets \(S\). In particular, the homotopy category of
  \(\mathbf M_{\leq 0}\) is equivalent to the category \((\mathbf
  M_{\leq 0})^{cf}\) and we will denote it by \(\disc(\mathbf M)\).

  As a matter of terminology, given an object \(X\in\mathbf M\),
  homotopy types in \(\disc(\mathbf M/X)\) will be called
  \emph{sheaves on \(X\)}.
\end{remark}

By functorial factorizations,
\autoref{thm:truncated-model-structure-exists} implies, for every
\(k\geq -2\), the existence of fibrant replacement functors
\[
\trunc{k}\colon \mathbf{M}\to\mathbf{M}|_{\leq k}
\]
for \(\mathbf M_{\leq k}\). The functor \(\trunc{k}\) is called the
\emph{\(k\textsuperscript{th}\) truncation functor} for
\(\mathbf{M}\). In particular, given any object \(X\in\mathbf{M}\) we
get maps
\[
X\to \trunc{k} X\to *,
\]
the first one being an acyclic cofibration in \(\mathbf{M}_{\leq k}\)
and the second one being a fibration in \(\mathbf{M}_{\leq
  k}\). \Autoref{cor:weak-equiv-level} implies that there are natural
weak equivalences
\[
\RHom_{\mathbf M_{\leq k}}(\trunc{k} X, Y)\simeq \RHom_{\mathbf
  M}(\trunc{k} X, Y) \simeq \RHom_{\mathbf M}(X, Y)
\]
whenever \(Y\) is \(k\)--truncated. Concretely, every morphism
\(f\colon X\to Y\) factors up to weak equivalence as morphisms \(X\to
\trunc{k}X\to Y\) if \(Y\) is \(k\)--truncated.

\begin{corollary}
  Given \(n \leq m\), for any \(X\in\mathbf M\) there is a natural
  weak equivalence \(\trunc{n}\trunc{m} X \simeq \trunc{n} X\) in
  \(\mathbf M\). In particular, \(\trunc{n}X\) is weakly equivalent to
  the \(n\textsuperscript{th}\) truncation of \(\tau_{m}X\) for every
  \(m\geq n\).
\end{corollary}
\begin{proof}
  Observe that \(n\leq m\) implies that \(X \simeq_{\leq
    n}\trunc{m}X\), since \(\mathbf M|_{\leq n}\subset \mathbf
  M|_{\leq m}\) and \(X\simeq_{\leq m}\trunc{m}X\). Hence, given any
  \(Y\in\mathbf M\), we find
  \begin{align*}
    \RHom_{\mathbf M}(Y, \trunc{n}\trunc{m} X)&\simeq \RHom_{\mathbf M}(\trunc{n}Y, \trunc{n}\trunc{m}X)\simeq \RHom_{\mathbf M_{\leq n}}(Y, \trunc{m}X)\simeq \\
    &\simeq \RHom_{\mathbf M_{\leq n}}(Y, X) \simeq \RHom_{\mathbf M}(\trunc{n}Y, \trunc{n}X)\simeq \RHom_{\mathbf M}(Y, \trunc{n}X).
  \end{align*}
  It follows that \(\trunc{n}\trunc{m}X\simeq \trunc{n}X\).
\end{proof}

For \(X\in\mathbf{M}\) a fixed object, the slice category \(\mathbf
M/X\) is again a combinatorial simplicial model category. Hence, by
the above, it has truncation functors which we will denote by
\(\trunc{k}^{X}\). Observe that an object \(f\colon Y\to X\) in
\(\mathbf M/X\) is \(k\)--truncated if and only if \(f\) is a
\(k\)--truncated morphism in \(\mathbf M\). Indeed, for any \(g\colon
Z\to X\) we have a weak equivalence
\[
\RHom_{\mathbf M/X}(g, f)\simeq \RHom_{\mathbf M}(Y, X)\times^{h}_{\RHom_{\mathbf M}(Z, X)} \{g\}.
\]

\begin{proposition}\label{prop:left-exact-preserves-trunc}
  Let \(\mathbf M\) and \(\mathbf N\) be combinatorial simplicial
  model categories and let
  \[
  \begin{tikzcd}
    \mathbf M \arrow[r, shift right=.7ex, "G"'] & \arrow[l, shift right=.7ex, "F"'] \mathbf N
  \end{tikzcd}
  \]
  be a simplicial Quillen adjunction. Assume that the derived left adjoint \(\LL F\) preserves finite homotopy limits. Then there is a natural weak equivalence
  \[\trunc{n}^{\mathbf M} \LL F(X) \simeq \LL F(\trunc{n}^{\mathbf N} X)\]
  for every \(X\in\mathbf N\).
\end{proposition}
\begin{proof}
  By \autoref{prop:truncated-objects-exact-functors} \(\LL F\) sends
  \(n\)--truncated objects in \(\mathbf N\) to \(n\)--truncated
  objects in \(\mathbf M\). Hence, the canonical morphism \(\LL
  F(X)\to\LL F(\trunc{n}^{\mathbf N}X)\) factors up to weak
  equivalence through a morphism \(f\colon \trunc{n}^{\mathbf M}\LL
  F(X)\to \LL F(\trunc{n}^{\mathbf N}X)\). To show that this is a weak
  equivalence, let \(Y\in\mathbf M\) be \(n\)--truncated. Then the
  induced morphism \(f^{*}\) factors through weak equivalences
  \begin{align*}
    \RHom_{\mathbf M}(\LL F(\trunc{n}^{\mathbf N}X), Y)&\simeq \RHom_{\mathbf N}(\trunc{n}^{\mathbf N}X, \RR G(Y)) \simeq\\
                                                    &\simeq \RHom_{\mathbf N}(X, \RR G(Y)) \simeq \\
                                                    &\simeq \RHom_{\mathbf M}(\LL F(X), Y) \simeq \\
                                                    &\simeq \RHom_{\mathbf M}(\trunc{n}^{\mathbf M}\LL F(X), Y).
  \end{align*}
  Hence, \(f\) is \(n\)--connected. But since \(f\) is a morphism
  beteween \(n\)--truncated objects, this is enough to show that \(f\)
  is a weak equivalence in \(\mathbf M\).
\end{proof}

\section{Homotopy Sheaves}

Given a space \(X\) in classical homotopy theory one can always choose
a basepoint \(x\in X\) and define homotopy groups \(\pi_{n}(X,x) =
[S^{n}, (X,x)]_{*} = \pi_{0}\RHom_{*}(S^{n}, (X,x))\). For \(n\geq 1\)
the group structure comes from the usual (homotopy) ogroup structure
on \(S^{n}\): the multiplication is induced by precomposition with the
\enquote{pinch map} \(S^{n}\to S^{n}\vee S^{n}\).

The main difficulty in working with homotopy groups in more general
homotopy theories is that for various purposes restricting one's
attention to pointed spaces is not good enough. This phenomenon is
entirely analogous to the observation that a sheaf of sets on a
topological space need not have any global section. For example, the
sheaf of local sections of a torsor on a topological space has a
global section if and only if the torsor is trivial. One could now
imagine working with a more general notion of basepoint for objects in
an arbitrary model category; for example, for simplicial sheaves there
certainly exist sections over small enough open sets. But doing so
amounts to choosing a presentation of a combinatorial model category
in the sense of \cite{mr1870516}.

In light of these considerations we will attempt to treat all possible
basepoints at the same time. Let us first spell out the case of
simplicial sets. Let \(X\) be some simplicial set (connected or
not). Choose some simplicial model \(S^{n}\) for the
\(n\)--sphere. Then we can consider the simplicial set \(X^{S^{n}}\)
of all simplicial maps \(S^{n}\to X\). Evaluation at the canonical
basepoint of \(S^{n}\) induces a map \(p\colon X^{S^{n}}\to
X\). Observe that the homotopy fiber of \(p\) over some point \(x\in
X\) is precisely \(\RHom_{*}(S^{n}, (X,x))\), the space of
\emph{pointed} maps from \(S^{n}\) to \((X,x)\). So the set of
connected components of this homotopy fiber is precisely
\(\pi_{n}(X,x)\). In this sense, the \(0\)--truncation of the
canonical map \(p\colon X^{S^{n}}\to X\) captures the
\(n\textsuperscript{th}\) homotopy groups of \(X\) at all possible
basepoints. This motivates the definition of homotopy sheaves in any
combinatorial simplicial model category.

\begin{definition}
  Let \(\mathbf M\) be a combinatorial simplicial model category and
  let \(X\in\mathbf M\). Then the \emph{\(n\textsuperscript{th}\)
    homotopy sheaf} of \(X\) is the homotopy type
  \(\pi_{n}(X)\coloneqq\trunc{0}^{X}(X^{S^{n}})\in\disc(\mathbf
  M/X)\).
\end{definition}

The sheaves \(\pi_{n}(X)\) admit canonical sections. Indeed, the
unique map \(S^{n}\to *\) induces a map \(X\simeq X^{*}\to X^{S^{n}}\)
which is a section of the map \(X^{S^{n}}\to X\). Applying
\(\trunc{0}^{X}\) yields a section of \(\pi_{n}(X)\). Hence, the
homotopy sheaves of \(X\) can be considered as pointed objects in
\(\disc(\mathbf M/X)\). However, to construct a group structure on
higher homotopy sheaves we will need to assume some exactness
properties for \(\mathbf M\); namely, \(\mathbf M\) needs to be a
\emph{model topos} in the sense of \cite{rezkhomotopytoposes}.

\begin{definition}\label{defn:model-topos}
  A combinatorial simplicial model category \(\mathbf M\) is a
  \emph{model topos} if there is a presentation of \(\mathbf M\) as a
  \emph{left exact} Bousfield localization of a category of simplicial
  presheaves. That is, \(\mathbf M\) is a model topos if there is a
  simplicial Quillen adjunction\footnote{Written in the direction of
    the right adjoint.}
  \[
  \begin{tikzcd}
    \mathbf M \arrow[r, shift right=.7ex, "i"'] & \arrow[l, shift right=.7ex, "L"'] \spshv(\mathbf C)
  \end{tikzcd}
  \]
  for some simplicial category \(\mathbf C\) and the injective model
  structure on \(\spshv(\mathbf C)\) such that \(\LL L\) preserves
  finite homotopy limits and \(\RR i\) is \emph{homotopically fully
    faithful}, i.\,e.~it induces weak equivalences
  \[
  \RHom(X, Y)\simeq\RHom(\RR i(X), \RR i(Y))
  \]
  for all \(X,Y\in\mathbf M\).
\end{definition}

Rezk proves in \cite[Corollary~6.10]{rezkhomotopytoposes} that for
\(X\) an object in a model topos \(\mathbf M\), the slice category
\(\mathbf M/X\) is again a model topos.

\begin{lemma}\label{lem:trunc-presheaves}
  Let \(\mathbf C\) be a simplicial category and let
  \(F\in\spshv(\mathbf C)\) be a simplicial presheaf. Then, for every
  \(c\in\mathbf C\) there is a natural weak equivalence
  \((\trunc{n}F)(c)\simeq \trunc{n}F(c)\) of simplicial sets.
\end{lemma}
\begin{proof}
  This follows from the fact that homotopy limits in \(\spshv(\mathbf
  C)\) may be computed pointwise and
  \autoref{prop:truncated-objects-exact-functors}.
\end{proof}

\begin{proposition}[{see \cite[Lemma~6.5.1.2]{mr2522659}}]\label{prop:trunc-products}
  If \(\mathbf M\) is a model topos, then the truncation functors
  \(\trunc{n}\) preserve finite homotopy products.
\end{proposition}
\begin{proof}
  First, let \(F,G\in\spshv(\mathbf C)\) be simplicial presheaves on
  some simplicial category \(\mathbf C\) and consider the canonical
  morphism
  \[
  \trunc{n}(F\times G)\to \trunc{n}F\times\trunc{n}G.
  \]
  To prove that it is a weak equivalence, it will be enough to check
  this after evaluating at all \(c\in\mathbf C\). Hence, by
  \autoref{lem:trunc-presheaves} we have reduced to the case of
  simplicial sets. There it follows from Whitehead's theorem by
  computing homotopy groups of both sides.

  Now choose a simplicial Quillen adjunction
  \[
  \begin{tikzcd}
    \mathbf M \arrow[r, shift right=.7ex, "i"'] & \arrow[l, shift right=.7ex, "L"'] \spshv(\mathbf C)
  \end{tikzcd}
  \]
  as in \autoref{defn:model-topos} and take \(X, Y\in\mathbf M\). By
  our previous considerations we have a natural weak equivalence
  \(\trunc{n}(\RR i(X)\times^{h}\RR i(Y))\simeq \trunc{n}\RR
  i(X)\times^{h}\trunc{n}\RR i(Y)\). Since \(\RR i\) is homotopically
  fully faithful the counit of the adjunction \(\LL L\dashv \RR i\)
  induces a weak equivalence \(\LL L\,\RR i (Z)\simeq Z\) for all
  \(Z\). Consequently, \autoref{prop:left-exact-preserves-trunc}
  provides us with weak equivalences
  \begin{align*}
    \trunc{n}(X\times^{h} Y) &\simeq \trunc{n}(\LL L\,\RR i(X\times^{h} Y)) \simeq \LL L\trunc{n}(\RR i(X)\times^{h} \RR i(Y)) \simeq \\
                         &\simeq \LL L(\trunc{n}\RR i(X)\times^{h} \trunc{n}\RR i(Y)) \simeq \\
                         &\simeq \trunc{n}(\LL L\,\RR i(X)) \times^{h} \trunc{n}(\LL L\,\RR i(Y)) \simeq \\
                         &\simeq \trunc{n}(X) \times^{h} \trunc{n}(Y)
  \end{align*}
  since \(\LL L\) is assumed to preserve finite homotopy limits.
\end{proof}

\Autoref{prop:trunc-products} allows us to produce a group structure
on \(\pi_{n}(X)\) for \(n\geq 1\) if \(X\) is an object in a model
topos \(\mathbf M\): We have the usual (homotopy) cogroup structure on
\(S^{n}\) given by the pinch map \(S^{n}\to S^{n}\vee S^{n}\). This
induces a morphism
\[
X^{S^{n}}\times^{h}_{X} X^{S^{n}} \simeq X^{S^{n}}\times^{h}_{X^{*}}
X^{S^{n}} \simeq X^{S^{n}\vee S^{n}} \to X^{S^{n}}.
\]
By \autoref{prop:trunc-products} the functor \(\trunc{0}^{X}\)
preserves finite homotopy products in \(\mathbf M/X\), so we get a
morphism
\[
\pi_{n}(X)\times_{X}\pi_{n}(X)\isom
\trunc{0}^{X}(X^{S^{n}})\times_{X}\trunc{0}^{X}(X^{S^{n}})\isom
\trunc{0}^{X}(X^{S^{n}\vee S^{n}}) \to \trunc{0}^{X}(X^{S^{n}}) \isom
\pi_{n}(X)
\]
in \(\disc(\mathbf M/X)\). It is then straightforward to check that
this gives group objects \(\pi_{n}(X)\) in \(\disc(\mathbf M/X)\),
which are abelian for \(n\geq 2\).

Given a map \(f\colon X\to Y\) in \(\mathbf M\) we can consider it as
an object of \(\mathbf M/Y\). As such, \(f\) has homotopy sheaves
\(\pi_{n}(f)\in\disc((\mathbf M/Y)/f)\). There is an equivalence of
categories \((\mathbf M/Y)/f\simeq \mathbf M/X\) which turns out to be
a simplicial Quillen equivalence. Hence, we can consider the homotopy
sheaves \(\pi_{n}(f)\) as objects of \(\disc(\mathbf M/X)\).

One can give a slightly more concrete description of \(\pi_{n}(f)\) as
follows. Consider the homotopy fiber product
\(X^{S^{n}}\times^{h}_{Y^{S^{n}}} Y\) along the canonical section
\(Y\to Y^{S^{n}}\). This is a model for the object \(f^{S^{n}}\in
\mathbf M/Y\). Hence, along the equivalence \((\mathbf M/Y)/f\simeq
\mathbf M/X\), we find that
\[
\trunc{0}^{X}(X^{S^{n}}\times^{h}_{Y^{S^{n}}} Y)\to X
\]
is a model for the homotopy sheaves \(\pi_{n}(f)\in\disc(\mathbf M/X)\).

\begin{remark}\label{rem:homotopy-group-functoriality}
  Rezk proves in \cite[Example~6.13]{rezkhomotopytoposes} that any
  morphism \(f\colon X\to Y\) in a model topos \(\mathbf M\) induces a
  simplicial Quillen adjunction
  \[
  \begin{tikzcd}
    \mathbf M/X \arrow[r, shift right=.7ex, "f_{*}"'] & \arrow[l, shift right=.7ex, "f^{*}"'] \mathbf M/Y
  \end{tikzcd}
  \]
  where the left adjoint \(f^{*}\) is given by homotopy pullback along
  \(f\) and is left exact. Consequently,
  \autoref{prop:left-exact-preserves-trunc} shows that \(f^{*}\)
  commutes with truncation. There is a homotopy commutative diagram
  \[
  \begin{tikzcd}
    X^{S^{n}} \arrow[r] \arrow[d] & f^{*}Y^{S^{n}} \arrow[r] \arrow[d] & Y^{S^{n}} \arrow[d] \\
    X \arrow[r, equals] & X \arrow[r, "f"'] & Y
  \end{tikzcd}
  \]
  and applying \(\trunc{0}^{X}\), we obtain a morphism
  \[
  f_{*}\colon \pi_{n}(X)\simeq\trunc{0}^{X}(X^{S^{n}})\longrightarrow
  \trunc{0}^{X}(f^{*}Y^{S^{n}})\simeq
  f^{*}\trunc{0}^{Y}(Y^{S^{n}})\simeq f^{*}\pi_{n}(Y).
  \]
\end{remark}

\begin{proposition}[{see~\cite[Remark~6.5.1.5]{mr2522659}}]
  Given maps \(f\colon X\to Y\) and \(g\colon Y\to Z\) in a model
  topos \(\mathbf M\), there is a long exact sequence
  \[
  \cdots \to f^{*}\pi_{n+1}(g) \to \pi_{n}(f) \to \pi_{n}(g\circ f) \to f^{*}\pi_{n}(g) \to \cdots
  \]
  of pointed objects in \(\disc(\mathbf M/X)\).
\end{proposition}
\begin{proof}
  We will first construct the relevant maps between homotopy
  sheaves. First, \(g\) induces a natural map
  \[
  g_{*}\colon X^{S^{n}}\times^{h}_{Y^{S^{n}}} Y\to
  X^{S^{n}}\times^{h}_{Z^{S^{n}}} Z
  \]
  and by functoriality of truncation we get at map \(g_{*}\colon
  \pi_{n}(f)\to\pi_{n}(g\circ f)\). Second, applying
  \autoref{rem:homotopy-group-functoriality} to the model topos
  \(\mathbf M/Z\) yields the required morphism \(f_{*}\colon
  \pi_{n}(g\circ f)\to f^{*}\pi_{n}(g)\).
  \viktor{finish constructing the connecting homomorphism and prove exactness}
\end{proof}


\section{Postnikov Towers}

Using truncation functors we can define Postnikov towers in any
combinatorial simplicial model category. Recall that a \emph{tower} in
\(\mathbf M\) is a diagram of the form
\[
\cdots \to X_{n+1}\to X_{n}\to X_{n-1}\to \cdots \to X_{0}
\]
and a \emph{convergent tower} in \(\mathbf M\) is a diagram
\[
X_{\infty}\xrightarrow{\pi} \cdots \to X_{n+1}\to X_{n}\to\cdots\to
X_{0}
\]
exhibiting \(X_{\infty}\) as the homotopy limit of the tower
\(\{X_{n}\}_{n\in\NN}\), i.\,e.~such that the morphism
\[
X_{\infty}\xrightarrow{\pi}\holim_{n\in\NN} X_{n}
\]
is a weak equivalence.

\begin{definition}
  Let \(\mathbf M\) be a combinatorial simplicial model category and
  \(X\in\mathbf M\). The tower
  \[
  \cdots\to\trunc{n} X\to\trunc{n-1} X\to\cdots\to\trunc{1}
  X\to\trunc{0}X
  \]
  is called the \emph{Postnikov tower} for \(X\). The Postnikov tower
  for \(X\) is said to \emph{converge} if
  \[
  X\to\cdots\to\trunc{n} X\to\trunc{n-1} X\to\cdots\to\trunc{1}
  X\to\trunc{0}X
  \]
  is a convergent tower in \(\mathbf M\).
\end{definition}

\printbibliography

\listoftodos
\end{document}
%%% Local Variables:
%%% mode: latex
%%% TeX-master: t
%%% TeX-command-default: "LatexMk"
%%% End:

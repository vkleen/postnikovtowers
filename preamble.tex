\RequirePackage[l2tabu, orthodox]{nag}
\documentclass[11pt,headsepline=true,toc=flat]{scrartcl}
\usepackage[utf8]{inputenc}

\usepackage{xspace, xifthen, enumitem}

\usepackage{amssymb, amsmath, amsthm, thmtools, nth}

\usepackage{graphicx}

\usepackage{etoolbox}

\usepackage{tikz}
\usetikzlibrary{arrows,decorations.markings,chains,calc,matrix,cd}
\tikzset{>=cm to}

\usepackage[lining]{libertine}
\usepackage[T1]{fontenc}
\usepackage{textcomp}
\usepackage[varqu,varl]{inconsolata}
\usepackage[italic, basic, eulergreek, defaultmathsizes]{mathastext}
\usepackage{bm}
\usepackage{mathtools}
\mathtoolsset{mathic}

\usepackage[cal=boondoxo]{mathalfa}
\usepackage{mathrsfs}
\newcommand{\mathcalls}[1]{{\textls*[-150]{\usefont{U}{BOONDOX-calo}{m}{n} #1}}}

\usepackage[colorlinks=true]{hyperref}
\usepackage[all]{hypcap}

\usepackage{csquotes}
\usepackage[english]{babel}
\usepackage[nodayofweek]{datetime}

\usepackage[protrusion=true]{microtype}

\usepackage[bibencoding=utf8,style=alphabetic,citestyle=alphabetic,backref=true,hyperref=true,giveninits=true,doi=true]{biblatex}
\addbibresource{all.bib}
\renewcommand*{\bibfont}{\normalfont\footnotesize}
\renewbibmacro{in:}{%
  \ifentrytype{article}{}{\printtext{\bibstring{in}\intitlepunct}}}
\renewrobustcmd*{\bibinitdelim}{\,}
\AtEveryBibitem{%
  \clearfield{pagetotal}%
}

\usepackage{pdfpages}

\declaretheoremstyle[spaceabove=\topsep,spacebelow=\topsep,headfont=\normalfont\scshape,notefont=\normalfont\mdseries,notebraces={(}{)},bodyfont=\normalfont,postheadspace=5pt plus 1pt minus 1pt]{scdef}
\declaretheoremstyle[spaceabove=\topsep,spacebelow=\topsep,headfont=\normalfont\scshape,notefont=\normalfont\mdseries,notebraces={(}{)},bodyfont=\itshape,postheadspace=5pt plus 1pt minus 1pt]{scthm}

\declaretheorem[style=scdef,numberwithin=section,   name=Definition,refname={Definition,Definitions},Refname={Definition,Definitions}]{definition}
\declaretheorem[style=scdef,sharenumber=definition, name=Remark,refname={Remark,Remarks},Refname={Remark,Remarks}]{remark}
\declaretheorem[style=scdef,sharenumber=definition, name=Example,refname={Example,Examples},Refname={Example,Examples}]{example}

\declaretheorem[style=scthm,sharenumber=definition, name=Theorem,refname={Theorem,Theorems},Refname={Theorem,Theorems}]{theorem}
\declaretheorem[style=scthm,sharenumber=definition, name=Lemma,refname={Lemma,Lemmas},Refname={Lemma,Lemmas}]{lemma}
\declaretheorem[style=scthm,sharenumber=definition, name=Corollary,refname={Corollary,Corollaries},Refname={Corollary,Corollaries}]{corollary}
\declaretheorem[style=scthm,sharenumber=definition, name=Proposition,refname={Proposition,Propositions},Refname={Proposition,Propositions}]{proposition}

\undef\Re
\undef\Im

\newcommand*{\normal}{\lhd}
\newcommand*{\isom}{\cong}
\newcommand*{\homot}{\sim}
\newcommand*{\wequiv}{\simeq}

\makeatletter
\let\@oldsubset=\subset
\def\@subsethelper#1#2{\mathrel{\raisebox{.5pt}{$#1\@oldsubset$}}\xspace}
\DeclareRobustCommand*{\subset}{\mathpalette\@subsethelper\relax}

\let\@oldotimes=\otimes
\def\@otimeshelper#1#2{\mathrel{\raisebox{.5pt}{$#1\@oldotimes$}}\xspace}
\DeclareRobustCommand*{\otimes}{\mathpalette\@otimeshelper\relax}
\makeatother

\tikzset{/tikz/commutative diagrams/arrows={thin}}

\newcommand*{\ca}[1]{\ensuremath{\mathscr{#1}}\xspace}
\renewcommand*{\cal}[1]{\ensuremath{\mathcal{#1}}\xspace}
\newcommand*{\f}[1]{\ensuremath{\mathfrak{#1}}\xspace}

\newcommand{\cl}[2][0]{{}\mkern#1mu\overline{\mkern-#1mu#2}}
\newcommand*{\Int}[1]{\ensuremath{#1^\circ}\xspace}

\newcommand*{\ie}{i.\,e.}
\newcommand*{\eg}{e.\,g.}
\newcommand*{\Ie}{I.\,e.}
\newcommand*{\Eg}{E.\,g.}

\undef\lrcorner
\newcommand{\lrcorner}{\mathord{\vrule height 0.1ex depth 0pt width 1ex\vrule height 1.3ex depth 0pt width
0.1ex}}

\def\<#1>{\left\langle #1 \right\rangle}

\undef\AA
\undef\SS
\renewcommand*{\do}[1]{\expandafter\def\csname#1#1\endcsname{\ensuremath{\mathbb{#1}}\xspace}}
\docsvlist{A,B,C,D,E,F,G,H,I,J,K,L,M,N,O,P,Q,R,S,T,U,V,W,X,Y,Z}

\let\to\longrightarrow

\setlist[enumerate]{label={\normalfont \rmfamily(\roman*)}, nosep}
\setlist[itemize]{nosep}

\overfullrule=1mm

\usepackage{todonotes}
\newcommand{\aravind}[1]{\todo[color=red!40]{#1}} %notes by Aravind
\newcommand{\marc}[1]{\todo[color=blue!40]{#1}} %notes by Marc
\newcommand{\matthias}[1]{\todo[color=cyan!40]{#1}} %notes by Matthias
\newcommand{\brad}[1]{\todo[color=magenta!40]{#1}} %notes by Brad
\newcommand{\viktor}[1]{\todo[color=yellow!40]{#1}} %notes by Viktor

\undef\lim
\DeclareMathOperator*{\lim}{\textnormal{colim}}
\DeclareMathOperator*{\colim}{\textnormal{colim}}
\DeclareMathOperator*{\hocolim}{\textnormal{hocolim}}
\DeclareMathOperator*{\holim}{\textnormal{holim}}

\DeclareMathOperator{\sk}{\textnormal{sk}}
\DeclareMathOperator{\cosk}{\textnormal{cosk}}

\undef\hom
\DeclareMathOperator{\hom}{\textnormal{Hom}}
\DeclareMathOperator{\Hom}{\textnormal{\bfseries Hom}}
\DeclareMathOperator{\RHom}{\mathbb{R}\textnormal{\bfseries Hom}}

\DeclareMathOperator{\fib}{\textnormal{fib}}
\DeclareMathOperator{\cof}{\textnormal{cof}}

\newcommand{\trunc}[1]{\tau_{#1}}

\makeatletter
\newbox\@xrat
\renewcommand*{\xrightarrow}[2][-cm to]{%
  \setbox\@xrat=\hbox{\ensuremath{\scriptstyle #2}}
  \pgfmathsetlengthmacro{\@xratlen}{max(1.6em,\wd\@xrat+.6em)}
  \pgfmathsetlengthmacro{\@xratinnersep}{.5ex-\dp\@xrat}
  \mathrel{\tikz [#1,baseline=-.6ex]
    \draw (0,0) -- node[auto,inner sep=\@xratinnersep] {\box\@xrat} (\@xratlen,0) ;}}
\renewcommand*{\xleftarrow}[2][cm to-]{%
  \setbox\@xrat=\hbox{\ensuremath{\scriptstyle #2}}
  \pgfmathsetlengthmacro{\@xratlen}{max(1.6em,\wd\@xrat+.6em)}
  \pgfmathsetlengthmacro{\@xratinnersep}{.5ex-\dp\@xrat}
  \mathrel{\tikz [#1,baseline=-.6ex]
    \draw (0,0) -- node[auto,inner sep=\@xratinnersep] {\box\@xrat} (\@xratlen,0) ;}}
\newcommand*{\xrightarrowb}[2][-cm to]{%
  \setbox\@xrat=\hbox{\ensuremath{\scriptstyle #2}}
  \pgfmathsetlengthmacro{\@xratlen}{max(1.6em,\wd\@xrat+.6em)}
  \pgfmathsetlengthmacro{\@xratinnersep}{.5ex}
  \mathrel{\tikz [#1,baseline=-.6ex]
    \draw (0,0) -- node[auto,inner sep=\@xratinnersep] {\box\@xrat} (\@xratlen,0) ;}}
\newcommand*{\xleftarrowb}[2][cm to-]{%
  \setbox\@xrat=\hbox{\ensuremath{\scriptstyle #2}}
  \pgfmathsetlengthmacro{\@xratlen}{max(1.6em,\wd\@xrat+.6em)}
  \pgfmathsetlengthmacro{\@xratinnersep}{.5ex}
  \mathrel{\tikz [#1,baseline=-.6ex]
    \draw (0,0) -- node[auto,inner sep=\@xratinnersep] {\box\@xrat} (\@xratlen,0) ;}}

\pgfarrowsdeclare{my right hook}{my right hook}
{
\arrowsize=0.2pt
\advance\arrowsize by .5\pgflinewidth
\pgfarrowsleftextend{-.5\pgflinewidth}
\pgfarrowsrightextend{3.5\arrowsize+.5\pgflinewidth}
}
{
\arrowsize=0.2pt
\advance\arrowsize by .5\pgflinewidth
\pgfsetdash{}{0pt} % do not dash
\pgfsetroundjoin % fix join
\pgfsetroundcap % fix cap
\pgfpathmoveto{\pgfpoint{0\arrowsize}{-7\arrowsize}}
\pgfpatharc{-90}{90}{3.5\arrowsize}
\pgfusepathqstroke
}

\tikzset{%
  iso/.style={above,sloped,inner sep=0},
  iso'/.style={below,sloped,inner sep=0},
  to/.style={-cm to},
  from/.style={cm to-},
  onto/.style={-cm double to},
  into/.style={my right hook-cm to},
  mapsto/.style={|-cm to},
  clim/.style={decoration={markings,
                           mark=at position#1 with {\draw[-] (0,-3\pgflinewidth) -- (0,3\pgflinewidth);}},
               postaction=decorate},
  clim/.default=0.5,
  opim/.style={decoration={markings,
                           mark=at position#1 with {\draw[-] circle(3\pgflinewidth);}},
               postaction=decorate},
  opim/.default=0.5
}

\newcommand*\@tikzto[2]{\begin{tikzpicture}[baseline]%
      \draw[to,line width={#2\pgflinewidth},scale=#1](0,.55ex) -- (1.6em,.55ex);%
    \end{tikzpicture}}

\newcommand*\@tikzfrom[2]{\begin{tikzpicture}[baseline]%
      \draw[from,line width={#2\pgflinewidth},scale=#1](0,.55ex) -- (1.6em,.55ex);%
    \end{tikzpicture}}

\newcommand*\@tikzcto[2]{\mathrel{\begin{tikzpicture}[baseline]%
      \draw[to,line width={#2\pgflinewidth},scale=#1](0,.55ex) -- (0.8em,.55ex);%
    \end{tikzpicture}}}

\newcommand*\@tikzonto[2]{\mathrel{\begin{tikzpicture}[baseline]%
      \draw[onto,line width={#2\pgflinewidth},scale=#1](0,.55ex) -- (1.6em,.55ex);%
    \end{tikzpicture}}}

\newcommand*\@tikzinto[2]{\mathrel{\begin{tikzpicture}[baseline]%
      \draw[into,line width={#2\pgflinewidth},scale=#1](0,.55ex) -- (1.6em,.55ex);%
    \end{tikzpicture}}}

\newcommand*\@tikzclim[2]{\mathrel{\begin{tikzpicture}[baseline]%
      \draw[into,clim,line width={#2\pgflinewidth},scale=#1](0,.55ex) -- (1.6em,.55ex);%
    \end{tikzpicture}}}

\newcommand*\@tikzopim[2]{\mathrel{\begin{tikzpicture}[baseline]%
      \draw[into,opim,line width={#2\pgflinewidth},scale=#1](0,.55ex) -- (1.6em,.55ex);%
    \end{tikzpicture}}}

\newcommand*\@tikzmapsto[2]{\begin{tikzpicture}[baseline]%
      \draw[mapsto,line width={#2\pgflinewidth},scale=#1](0,.55ex) -- (1.6em,.55ex);%
    \end{tikzpicture}}

\newcommand*\@tikziso[4]{\mathrel{\begin{tikzpicture}[baseline]%
      \draw[to,line width={#2\pgflinewidth},scale=#1](0,.55ex) -- node[iso,pos=0.47,inner sep=#4]{$#3\sim$} (1.6em,.55ex);%
    \end{tikzpicture}}}

\newsavebox{\@todisplay}
\savebox{\@todisplay}{\@tikzto{1.0}{1}}

\newsavebox{\@totext}
\savebox{\@totext}{\@tikzto{1.0}{1}}

\newsavebox{\@toscript}
\savebox{\@toscript}{\@tikzto{0.8}{0.9}}

\newsavebox{\@toscriptscript}
\savebox{\@toscriptscript}{\@tikzto{0.8}{0.75}}

\newcommand*\tikzto{\mathrel{\mathchoice{\usebox{\@todisplay}}%
  {\usebox{\@totext}}%
  {\usebox{\@toscript}}%
  {\usebox{\@toscriptscript}}}}

\newsavebox{\@mapstodisplay}
\savebox{\@mapstodisplay}{\@tikzmapsto{1.0}{1}}

\newsavebox{\@mapstotext}
\savebox{\@mapstotext}{\@tikzmapsto{1.0}{1}}

\newsavebox{\@mapstoscript}
\savebox{\@mapstoscript}{\@tikzmapsto{0.8}{0.9}}

\newsavebox{\@mapstoscriptscript}
\savebox{\@mapstoscriptscript}{\@tikzmapsto{0.8}{0.75}}

\newcommand*\tikzmapsto{\mathrel{\mathchoice{\usebox{\@mapstodisplay}}%
  {\usebox{\@mapstotext}}%
  {\usebox{\@mapstoscript}}%
  {\usebox{\@mapstoscriptscript}}}}

\newcommand*\tikzfrom{\mathrel{\mathchoice{\@tikzfrom{1.0}{1}}{\@tikzfrom{1.0}{1}}{\@tikzfrom{0.8}{0.9}}{\@tikzfrom{0.6}{0.75}}}}
\newcommand*\tikzcto{\mathchoice{\@tikzcto{1.0}{1}}{\@tikzcto{1.0}{1}}{\@tikzcto{0.8}{0.9}}{\@tikzcto{0.6}{0.75}}}
\newcommand*\tikzonto{\mathchoice{\@tikzonto{1.0}{1}}{\@tikzonto{1.0}{1}}{\@tikzonto{0.8}{0.9}}{\@tikzonto{0.6}{0.75}}}
\newcommand*\tikzinto{\mathchoice{\@tikzinto{1.0}{1}}{\@tikzinto{1.0}{1}}{\@tikzinto{0.8}{0.9}}{\@tikzinto{0.6}{0.75}}}
\newcommand*\tikzclim{\mathchoice{\@tikzclim{1.0}{1}}{\@tikzclim{1.0}{1}}{\@tikzclim{0.8}{0.9}}{\@tikzclim{0.6}{0.75}}}
\newcommand*\tikzopim{\mathchoice{\@tikzopim{1.0}{1}}{\@tikzopim{1.0}{1}}{\@tikzopim{0.8}{0.9}}{\@tikzopim{0.6}{0.75}}}
\newcommand*\tikziso{\mathchoice{\@tikziso{1.0}{1}{\displaystyle}{0pt}}%
  {\@tikziso{1.0}{1}{\textstyle}{0pt}}%
  {\@tikziso{0.8}{0.9}{\scriptstyle}{0pt}}%
  {\@tikziso{0.67}{0.8}{\scriptscriptstyle}{0.15ex}}}
\makeatother

\renewcommand*{\to}[1][]{\ifthenelse{\isempty{#1}}{\tikzto}{\xrightarrowb{#1}}}
\newcommand*{\from}[1][]{\ifthenelse{\isempty{#1}}{\tikzfrom}{\xleftarrowb{#1}}}
\newcommand*{\cto}{\ensuremath{\tikzcto}}
\newcommand*{\into}[1][]{\ifthenelse{\isempty{#1}}{\tikzinto}{\xrightarrowb[into]{#1}}}
\newcommand*{\onto}[1][]{\ifthenelse{\isempty{#1}}{\tikzonto}{\xrightarrowb[onto]{#1}}}
\newcommand*{\clim}{\tikzclim}
\newcommand*{\opim}{\tikzopim}

\newcommand*{\iso}{\tikziso}

\renewcommand*{\mapsto}{\tikzmapsto}


\RequirePackage[l2tabu, orthodox]{nag}
\documentclass[11pt,headsepline=true,toc=flat]{scrartcl}

\usepackage{xspace, xifthen, enumitem}

\usepackage{amssymb, amsmath, amsthm, thmtools, nth}

\usepackage{graphicx}

\usepackage{etoolbox}

\usepackage{tikz}
\usetikzlibrary{arrows,decorations.markings,chains,calc,matrix}
\usepackage{tikz-cd}
\tikzset{>=cm to}

\usepackage[lining]{libertine}
\usepackage[T1]{fontenc}
\usepackage{textcomp}
\usepackage[varqu,varl]{inconsolata}
\usepackage[italic, basic, eulergreek, defaultmathsizes]{mathastext}
\usepackage{bm}
\usepackage{mathtools}
\mathtoolsset{mathic}

\usepackage[cal=boondoxo]{mathalfa}
\usepackage{mathrsfs}
\newcommand{\mathcalls}[1]{{\textls*[-150]{\usefont{U}{BOONDOX-calo}{m}{n} #1}}}

\usepackage[colorlinks=true]{hyperref}
\usepackage[all]{hypcap}

\usepackage{csquotes}
\usepackage[english]{babel}
\usepackage[nodayofweek]{datetime}

\usepackage[protrusion=true]{microtype}

\usepackage[bibencoding=utf8,style=alphabetic,citestyle=alphabetic,backref=true,hyperref=true,firstinits=true,doi=true]{biblatex}
\addbibresource{all.bib}
\renewcommand*{\bibfont}{\normalfont\footnotesize}
\renewbibmacro{in:}{%
  \ifentrytype{article}{}{\printtext{\bibstring{in}\intitlepunct}}}
\renewrobustcmd*{\bibinitdelim}{\,}
\AtEveryBibitem{%
  \clearfield{pagetotal}%
}

\usepackage{pdfpages}

\declaretheoremstyle[spaceabove=\topsep,spacebelow=\topsep,headfont=\normalfont\scshape,notefont=\normalfont\mdseries,notebraces={(}{)},bodyfont=\normalfont,postheadspace=5pt plus 1pt minus 1pt]{scdef}
\declaretheoremstyle[spaceabove=\topsep,spacebelow=\topsep,headfont=\normalfont\scshape,notefont=\normalfont\mdseries,notebraces={(}{)},bodyfont=\itshape,postheadspace=5pt plus 1pt minus 1pt]{scthm}

\declaretheorem[style=scdef,numberwithin=section,   name=Definition,refname={definition,definitions},Refname={Definition,Definitions}]{definition}
\declaretheorem[style=scdef,sharenumber=definition, name=Remark,refname={remark,remarks},Refname={Remark,Remarks}]{remark}
\declaretheorem[style=scdef,sharenumber=definition, name=Example,refname={example,examples},Refname={Example,Examples}]{example}

\declaretheorem[style=scthm,sharenumber=definition, name=Theorem,refname={theorem,theorems},Refname={Theorem,Theorems}]{theorem}
\declaretheorem[style=scthm,sharenumber=definition, name=Lemma,refname={lemma,lemmas},Refname={Lemma,Lemmas}]{lemma}
\declaretheorem[style=scthm,sharenumber=definition, name=Corollary,refname={corollary,corollaries},Refname={Corollary,Corollaries}]{corollary}
\declaretheorem[style=scthm,sharenumber=definition, name=Proposition,refname={proposition,propositions},Refname={Proposition,Propositions}]{proposition}

\undef\Re
\undef\Im

\newcommand*{\normal}{\lhd}
\newcommand*{\isom}{\cong}
\newcommand*{\homot}{\sim}
\newcommand*{\wequiv}{\simeq}

\makeatletter
\let\@oldsubset=\subset
\def\@subsethelper#1#2{\mathrel{\raisebox{.5pt}{$#1\@oldsubset$}}\xspace}
\DeclareRobustCommand*{\subset}{\mathpalette\@subsethelper\relax}

\let\@oldotimes=\otimes
\def\@otimeshelper#1#2{\mathrel{\raisebox{.5pt}{$#1\@oldotimes$}}\xspace}
\DeclareRobustCommand*{\otimes}{\mathpalette\@otimeshelper\relax}
\makeatother

\tikzset{/tikz/commutative diagrams/arrows={thin}}

\newcommand*{\ca}[1]{\ensuremath{\mathscr{#1}}\xspace}
\renewcommand*{\cal}[1]{\ensuremath{\mathcal{#1}}\xspace}
\newcommand*{\f}[1]{\ensuremath{\mathfrak{#1}}\xspace}

\newcommand{\cl}[2][0]{{}\mkern#1mu\overline{\mkern-#1mu#2}}
\newcommand*{\Int}[1]{\ensuremath{#1^\circ}\xspace}

\newcommand*{\ie}{i.\,e.}
\newcommand*{\eg}{e.\,g.}
\newcommand*{\Ie}{I.\,e.}
\newcommand*{\Eg}{E.\,g.}

\undef\lrcorner
\newcommand{\lrcorner}{\mathord{\vrule height 0.1ex depth 0pt width 1ex\vrule height 1.3ex depth 0pt width
0.1ex}}

\def\<#1>{\left\langle #1 \right\rangle}

\undef\AA
\undef\SS
\renewcommand*{\do}[1]{\expandafter\def\csname#1#1\endcsname{\ensuremath{\mathbb{#1}}\xspace}}
\docsvlist{A,B,C,D,E,F,G,H,I,J,K,L,M,N,O,P,Q,R,S,T,U,V,W,X,Y,Z}

\setlist[enumerate]{label={\normalfont \rmfamily(\roman*)}, nosep}
\setlist[itemize]{nosep}

\overfullrule=1mm

\usepackage{todonotes}
\newcommand{\aravind}[1]{\todo[color=red!40]{#1}} %notes by Aravind
\newcommand{\marc}[1]{\todo[color=blue!40]{#1}} %notes by Marc
\newcommand{\matthias}[1]{\todo[color=cyan!40]{#1}} %notes by Matthias
\newcommand{\brad}[1]{\todo[color=magenta!40]{#1}} %notes by Brad
\newcommand{\viktor}[1]{\todo[color=yellow!40]{#1}} %notes by Viktor

\undef\lim
\DeclareMathOperator*{\lim}{\textnormal{colim}}
\DeclareMathOperator*{\colim}{\textnormal{colim}}
\DeclareMathOperator*{\hocolim}{\textnormal{hocolim}}
\DeclareMathOperator*{\holim}{\textnormal{holim}}

\DeclareMathOperator{\sk}{\textnormal{sk}}
\DeclareMathOperator{\cosk}{\textnormal{cosk}}

\undef\hom
\DeclareMathOperator{\hom}{\textnormal{Hom}}
\DeclareMathOperator{\Hom}{\textnormal{\bfseries Hom}}
\DeclareMathOperator{\RHom}{\mathbb{R}\textnormal{\bfseries Hom}}

\DeclareMathOperator{\fib}{\textnormal{fib}}
\DeclareMathOperator{\cof}{\textnormal{cof}}

\newcommand{\trunc}[1]{\tau_{#1}}

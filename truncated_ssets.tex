\documentclass[main.tex]{subfiles}
\begin{document}

\section{Truncated and Connected Simplicial Sets}

\begin{definition}\label{defn:truncated_simp}
  A simplicial set \(X\) is called \emph{\(n\)--truncated} for \(n\geq-1\) if
  \(\pi_{i}(X,x) = *\) for every \(x\in X_{0}\) and every \(i>n\). A morphism
  \(f\colon Y\to X\) of simplicial sets is called \emph{\(n\)--truncated} if
  every homotopy fiber of \(f\) is \(n\)--truncated.

  By convention, a simplicial set is \((-2)\)--truncated if and only if it is
  contractible; a map of simplicial sets is \((-2)\)--truncated if it is a weak
  equivalence.
\end{definition}

\begin{lemma}\label{lem:truncated-diagonal}
  A simplicial set \(X\) is \(n\)--truncated if and only if the diagonal
  \(\Delta\colon X\to X\times X\) is \((n-1)\)--truncated.
\end{lemma}
\begin{proof}
  Let \(X\) be any connected simplicial set and let \((a,b)\in X\times X\). Let
  \(P\) be the homotopy fiber of \(\Delta\) at \((a,b)\). Any path from \(a\) to
  \(b\) in \(X\) induces a homotopy equivalence \(\Omega_{a}X \simeq P \simeq
  \Omega_{b} X\), hence isomorphisms \(\pi_{i+1}(X)\cong \pi_{i}(P)\). This
  proves the claim for connected simplicial sets. For the general case, write
  \(X = \bigsqcup_{j\in I} X_{j}\) as a disjoint union of its connected
  components and observe that \(X\times X = \bigsqcup_{j,\ell\in I} X_{j}\times
  X_{\ell}\). Then the homotopy fiber of the diagonal over a point \((a,b)\in
  X_{j}\times X_{\ell}\) is either empty (if \(j\neq \ell\)) or homotopy
  equivalent to \(\Omega_{a}X_{j}\) (if \(j = \ell\)). Since \(X\) is
  \(n\)--truncated if and only if all its connected components are
  \(n\)--truncated, this reduces the problem to the case of connected simplicial
  sets, which we have already dealt with.
\end{proof}

\begin{lemma}\label{lem:relative-truncated-diagonal}
  A map \(f\colon Y\to X\) of simplicial sets is \(n\)--truncated if and only if
  the diagonal \(\Delta_{f}\colon Y\to Y\times^{h}_{X} Y\) is
  \((n-1)\)--truncated.
\end{lemma}
\begin{proof}
  We can assume that \(f\) is a fibration between fibrant simplicial sets. If
  \(x\in X\) then the homotopy fiber of \(f\) is just the ordinary fiber product
  \(Y_{x}\coloneqq \{x\}\times_{X} Y\). Observe that \(\Delta_{f}\) is weakly
  equivalent to the fibration \(Y^{\Delta^{1}}\to Y\times_{X} Y\) and
  \(Y\times_{X}^{h} Y \simeq Y\times_{X} Y\). Given any \((y, y')\in Y\times_{X}
  Y\), let \(x = f(y) = f(y')\). Then there is a commutative diagram
  \[
    \begin{mytikzcd}
      \Omega_{y} Y_{x} \arrow[r] \arrow[d] & Y_{x}^{\Delta^{1}} \arrow[r] \arrow[d] & Y^{\Delta^{1}} \arrow[d] \\
      * \arrow[r, "{(y,y')}"'] & Y_{x}\times Y_{x} \arrow[r] & Y\times_{X} Y
    \end{mytikzcd}
  \]
  in which both small squares are cartesian. It follows that the homotopy fiber
  of \(\Delta_{f}\) at \((y,y')\) is weakly equivalent to the homotopy fiber of
  \(\Delta\colon Y_{x}\to Y_{x}\times Y_{x}\). Hence, all homotopy fibers of
  \(\Delta_{f}\) are \((n-1)\)--truncated if and only if all homotopy fibers of
  all diagonals \(Y_{x}\to Y_{x}\times Y_{x}\) are. By
  \autoref{lem:truncated-diagonal} this is the case if and only if all \(Y_{x}\)
  are \(n\)--truncated, \ie~if and only if the map \(Y\to X\) is
  \(n\)--truncated.
\end{proof}

\begin{corollary}\label{cor:truncatedness-makes-sense}
  A simplicial set \(X\) is \(n\)--truncated if and only if for every simplicial
  set \(Z\) the derived mapping space \(\RHom(Z, X)\) is \(n\)--truncated. A map
  \(f\colon X\to Y\) is \(n\)--truncated if and only if for every \(Z\) the map
  \(f_{*}\colon \RHom(Z, X)\to\RHom(Z, Y)\) is \(n\)--truncated.
\end{corollary}
\begin{proof}
  The proof proceeds by induction on \(n\). By
  \autoref{lem:relative-truncated-diagonal} a map \(f\colon Y\to X\) is
  \(n\)--truncated if and only if \(\Delta_{f}\colon Y\to Y\times_{X}^{h} Y\) is
  \((n-1)\)--truncated. By induction this is the case if and only if
  \(\RHom(Z,Y)\to \RHom(Z, Y\times_{X}^{h} Y)\) is \((n-1)\)--truncated. Observe
  that
  \[
    \begin{mytikzcd}
      \RHom(Z,Y) \arrow[r] \arrow[dr] & \RHom(Z, Y\times^{h}_{X} Y) \arrow[d, "\simeq"] \\
      {} & \RHom(Z, Y)\times_{\RHom(Z, X)}^{h} \RHom(Z,Y)
    \end{mytikzcd}
  \]
  commutes. Hence, \(\Delta_{f_{*}}\) is \((n-1)\)--truncated as well and again
  by \autoref{lem:relative-truncated-diagonal} this is equivalent to
  \(f_{*}\colon \RHom(Z, Y)\to\RHom(Z, X)\) being \(n\)--truncated.
\end{proof}

\begin{definition}\label{defn:connect_simp}
  Let \(n\geq 0\). A (nonempty) simplicial set \(X\) is called
  \emph{\(n\)--connected} if \(\pi_i(X,x) = *\) for every \(i\leq n\) and every
  \(x\in X_{0}\). A morphism \(f\colon Y\to X\) of simplicial sets is
  \emph{\(n\)--connected} or an \emph{\(n\)--equivalence} if all its homotopy
  fibers are \(n\)--connected.

  By convention, a simplicial set is \((-1)\)--connected if and only if it is
  nonempty and a map is a \((-1)\)--equivalence if it is surjective. Every
  simplicial set and every map of simplicial sets is \((-2)\)--connected.
\end{definition}

\begin{lemma}\label{lem:connected-homotopy-grps}
  Let \(n\geq 0\) and let \(f\colon Y\to X\) be a map of simplicial sets. Then
  \(f\) is \(n\)--connected if and only if, for every choice of \(y\in Y_{0}\),
  the induced map \(f_{*}\colon \pi_i(Y,y)\to\pi_{i}(X,f(y))\) is an isomorphism
  for \(i \leq n\) and an epimorphism for \(i = n+1\).
\end{lemma}
\begin{proof}
  Without loss of generality we can assume that \(f\) is a fibration between
  fibrant simplicial sets. Let \(y\in Y_{0}\) and write \(P = f^{-1}(f(y))\) for
  the fiber of \(f\) at \(f(y)\). The long exact sequence of \(f\) provides
  exact sequences
  \[
    \pi_{i+1}(P,y) \to \pi_{i+1}(Y, y) \to \pi_{i+1}(X,f(y)) \to \pi_{i}(P,y)
  \]
  for \(i\geq 0\) which proves the lemma.
\end{proof}

\begin{remark}
  \autoref{lem:connected-homotopy-grps} justifies the terminology
  \enquote{\(n\)--equivalence} for \(n\)--connected maps: A map \(f\colon Y\to
  X\) between \(n\)--truncated simplicial sets \(Y\) and \(X\) is a weak
  equivalence if and only if it is \(n\)--connected.
\end{remark}

\begin{definition}\label{defn:trunc_functor}
  A functor \(\trunc{n}\colon \sset \to \sset\) with a natural transformation
  \(\eta\colon \id\to\trunc{n}\) is an \emph{\Th{n} truncation functor} if
  \begin{itemize}
  \item \(\trunc{n}X\) is \(n\)--truncated for all \(X\in\sset\).
  \item the map \(\eta_X\colon X\to\trunc{n}X\) is an \(n\)--equivalence.
  \end{itemize}
\end{definition}

\viktor{Change notation of \(\sset_{\leq n}\) so as not to conflict with general
  notation \(\mathbf M_{\leq n}\),} We will now construct explicit truncation
functors for the category \(\sset\) of simplicial sets. First, the simplex
category \(\simplex\), when regarded as the category of nonzero finite ordinals,
has a natural filtration by full subcategories
\[
  * = \simplex_{\leq 0} \subset \simplex_{\leq 1} \subset\cdots\subset\simplex
\]
where \(\simplex_n\) is the category of nonzero finite ordinals less than or
equal to \(n\). This induces a tower of functors
\[
  \sset \to \cdots \to \sset_{\leq 1} \to \sset_{\leq 0} = \set
\]
and we write \(\iota_n^*\colon \sset \to \sset_{\leq n}\) for the restriction
functors.

By general abstract nonsense the functors \(\iota_n^*\) extend to adjunctions
\[
  (\iota_n)_! \dashv \iota_n^* \dashv (\iota_n)_*,
\]
see e.\,g.~\cite[Theorem~4, p.~59]{sheavesgeometrylogic} or \cite[{Exposé}~i,
Proposition~5.1]{SGA4-1}. Composing these, we get an adjunction
\[
  \sk_n = (\iota_n)_{!}\iota_n^{*} \dashv (\iota_n)_{*}\iota_n^{*} = \cosk_n.
\]

\begin{lemma}\label{lem:cosk_sk_composition}
  The counit is a natural isomorphism \(\id \simeq
  \iota_{n}^{*}(\iota_{n})_{!}\) and the unit is a natural isomorphism
  \(\iota_{n}^{*}(\iota_n)_{*}\simeq\id\). Consequently, there are induced
  natural isomorphisms \(\cosk_{n}\sk_{n}\simeq \cosk_{n}\) and
  \(\sk_n\cosk_n\simeq\sk_n\).
\end{lemma}
\begin{proof}
  By \cite[{Exposé}~i, Proposition~5.6]{SGA4-1} the assertion that
  \(\id\simeq\iota_n^{*}(\iota_n)_{!}\) and
  \(\iota_{n}^{*}(\iota_n)_{*}\simeq\id\) is equivalent to \(\iota_{n}\) being a
  fully faithful functor. But this is immediate from the definitions.

  For the second part, observe that
  \[
    \cosk_n\sk_n = (\iota_n)_{*}\iota_n^{*}(\iota_{n})_{!}\iota_n^{*} \simeq
    (\iota_n)_{*}\iota_n^{*} = \cosk_{n}
  \]
  and
  \[
    \sk_n\cosk_n = (\iota_n)_{!}\iota_n^{*}(\iota_n)_{*}\iota_n^{*}\simeq
    (\iota_n)_{!}\iota_n^{*}\simeq \sk_n.\qedhere
  \]
\end{proof}

We will abuse notation somewhat and simply call the composition
\(\id\to\cosk_{n}\sk_{n}\simeq\cosk_{n}\) the \emph{unit transformation} and the
composition \(\sk_{n}\simeq\sk_n\cosk_n\to\id\) the \emph{counit
  transformation}.

\begin{remark}\label{rem:unit-is-unit}
  In fact, the unit transformation \(\eta\colon
  \id\to\cosk_n=(\iota_n)_{*}\iota_n^{*}\) is the unit of the adjunction
  \(\iota_n^{*}\dashv (\iota_n)_{*}\): Writing \(\iota=\iota_n\), the
  isomorphism
  \[
    \hom(\iota_{!}\iota^{*}(X), \iota_{!}\iota^{*}(X)) \to \hom(X,
    \iota_{*}\iota^{*}\iota_{!}\iota^{*}(X))
  \]
  sends the identity map to
  \[
    \iota_{*}\Big(\iota^{*}({\id})\circ
    \unit^{\iota_{!}\dashv\iota^{*}}_{\iota^{*}(X)}\Big)\circ\unit_X^{\iota^{*}\dashv\iota_{*}}.
  \]
  Since the isomorphism
  \(\iota_{*}\iota^{*}(X)\to\iota_{*}\iota^{*}\iota_{!}\iota^{*}(X)\) is given
  by \(\iota_{*}(\unit^{\iota_{!}\dashv\iota^{*}}_{\iota^{*}(X)})\), it follows
  that the composite
  \[
    \eta_{X}\colon X \to \cosk_n\sk_n(X)\isom\cosk_n(X)
  \]
  is precisely \(\unit_X^{\iota^{*}\dashv\iota_{*}}\).

  Dually, the counit transformation \((\iota_n)_{!}\iota_{n}^{*} = \sk_n \to
  \id\) is the counit of the adjunction \((\iota_{n})_{!}\dashv\iota_{n}^{*}\).
\end{remark}

\begin{lemma}\label{lem:sk-comp}
  If \(n\leq i\), then there are natural isomorphisms \(\sk_n\sk_i\simeq \sk_n
  \simeq \sk_i\sk_n\).
\end{lemma}
\begin{proof}
  We have a commutative diagram of functors
  \[
    \begin{mytikzcd}[column sep=1em]
      {} & \simplex_{\leq i} \arrow[rd, "\iota_i"] & {} \\
      \simplex_{\leq n} \arrow[ru, "\iota_n^i"] \arrow[rr, "\iota_n"'] & &
      \simplex
    \end{mytikzcd}
  \]
  which induces commutative diagrams
  \[
    \begin{mytikzcd}[column sep=1em]
      {} & \sset_{\leq i} \arrow[dl, "(\iota_n^i)^{*}"'] & {} \\
      \sset_{\leq n} & & \sset \arrow[ul, "\iota_i^{*}"'] \arrow[ll,
      "\iota_n^{*}"]
    \end{mytikzcd}\quad\text{and}\quad
    \begin{mytikzcd}[column sep=1em]
      {} & \sset_{\leq i} \arrow[dr, "(\iota_i)_{!}"] & {} \\
      \sset_{\leq n} \arrow[ru, "(\iota_n^i)_{!}"] \arrow[rr, "(\iota_n)_{!}"']
      & & \sset.
    \end{mytikzcd}
  \]
  By \autoref{lem:cosk_sk_composition} we find
  \begin{align*}
    \sk_n\sk_i &= (\iota_n)_{!}\iota_n^{*}(\iota_i)_{!}\iota_i^{*} \simeq
                 (\iota_n)_{!}(\iota_n^i)^{*}\iota_i^{*}(\iota_i)_{!}\iota_i^{*} \simeq
    \\
               &\simeq (\iota_n)_{!}(\iota_{n}^i)^{*}\iota_i^{*}\simeq
                 (\iota_n)_{!}\iota_n^{*} = \sk_{n} \\
    \shortintertext{and}
    \sk_i\sk_n &= (\iota_i)_{!}\iota_i^{*}(\iota_n)_{!}\iota_n^{*}
                 \simeq
                 (\iota_i)_{!}\iota_i^{*}(\iota_i)_{!}(\iota_n^i)_{!}\iota_n^{*}
                 \simeq \\
               &\simeq (\iota_i)_{!}(\iota_n^i)_{!}\iota_n^{*} \simeq
                 (\iota_n)_{!}\iota_n^{*} = \sk_n.\qedhere
  \end{align*}
\end{proof}

\begin{proposition}\label{prop:cosk-is-truncated}
  If \(X\in\sset\) is a Kan complex, then \(\cosk_n X\) is an
  \((n-1)\)--truncated Kan complex.
\end{proposition}
\begin{proof}
  To show that \(\cosk_n X\) is a Kan complex, we need to show that any horn
  \(\Lambda^k_i\to\cosk_n X\) extends to a simplex \(\Delta^k\to\cosk_n X\) in
  \(\cosk_n X\). By adjunction, this is equivalent to showing that in any
  diagram
  \[
    \begin{mytikzcd}
      \sk_n \Lambda^k_i \arrow[r, "h"] \arrow[d, into] & X \\
      \sk_n \Delta^k \arrow[ur, dashed]
    \end{mytikzcd}
  \]
  a dashed arrow making the diagram commute exists. Now, if \(n<k-1\), then the
  horn inclusion \(\Lambda^k_i\into \Delta^k\) induces an isomorphism on
  \(\sk_n\), so the required extension exists trivially.

  If \(n \geq k-1\), then the counit is an isomorphism
  \(\sk_{k-1}\Lambda^k_i\isom \Lambda^k_i\) and by fibrancy of \(X\) there is an
  extension of \(h\) to a simplex \(\Delta^{k}\to X\). Naturality of the counit
  transformation yields a commutative square
  \[
    \begin{mytikzcd}
      \sk_n\Lambda^k_i \arrow[r, "{\isom}"] \arrow[d] & \Lambda^k_i
      \arrow[d, into] \\
      \sk_n\Delta^k \arrow[r] & \Delta^k
    \end{mytikzcd}
  \]
  which implies that we can take the composite \(\sk_n\Delta^k\to\Delta^k\to X\)
  as the required extension.

  To show that \(\cosk_n X\) is \((n-1)\)--truncated, let \(i\geq n\), choose
  some base point \(x\in (\cosk_n X)_{0}\) and let \([\alpha]\in\pi_i(\cosk_nX,
  x)\) be some homotopy class. Because \(\sk_{i}\Delta^{i+1}\) is a model for
  the \(i\)--sphere, we can take \(\alpha\) to be a map
  \(\sk_i\Delta^{i+1}\to\cosk_n X\). To prove that
  \([\alpha]=*\in\pi_i(\cosk_{n}X, x)\), it will then be enough to show that in
  \[
    \begin{mytikzcd}
      \sk_i\Delta^{i+1} \arrow[r, "\alpha"] \arrow[d] & \cosk_nX \\
      \Delta^{i+1} \arrow[ru, dashed]
    \end{mytikzcd}
  \]
  a dashed map making the diagram commute exists. By adjunction this is
  equivalent to showing that a dashed map in
  \[
    \begin{mytikzcd}
      \sk_n\sk_i\Delta^{i+1} \arrow[r] \arrow[d] & X \\
      \sk_n\Delta^{i+1} \arrow[ru, dashed]
    \end{mytikzcd}
  \]
  exists. But by \autoref{lem:sk-comp} there is a natural isomorphism
  \(\sk_n\sk_i\Delta^{i+1}\isom \sk_n\Delta^{i+1}\), which implies that such an
  extension always exists.
\end{proof}

\begin{definition}
  A simplicial set \(X\) is called \(n\)--skeletal if \(X\isom \sk_n X\).
  Equivalently, by \autoref{lem:sk-comp}, \(X\) is \(n\)--skeletal if and only
  if it lies in the essential image of \(\sk_{n}\).
\end{definition}

Note that, by \autoref{lem:sk-comp}, an \(n\)--skeletal simplicial set is
automatically \(i\)--skeletal for any \(i\leq n\). Also, since \(\sk_{n}\) is a
left adjoint, the property of being \(n\)--skeletal is preserved under arbitrary
colimits.

\begin{lemma}\label{lem:skeletal-simp-sets}
  Let \(X\) be any simplicial set and \(n\geq 0\).
  \begin{enumerate}
  \item If \(K\) is an \(n\)--skeletal simplicial set, then \(K\times\Delta^1\)
    is \((n+1)\)--skeletal.
  \item The functors \(\hom(\_,X)\) and \(\hom(\_, \cosk_nX)\) agree on the full
    subcategory of \(\sset\) consisting of \(n\)--skeletal simplicial sets, that
    is, composition with \(\eta_X\) is a natural isomorphism
    \[
      (\eta_X)_{*}\colon\hom(K,X)\to\hom(K,\cosk_nX)
    \]
    for \(n\)--skeletal \(K\).
  \item If \(L\) is an \((n-1)\)--skeletal simplicial set, \(X\) is a Kan
    complex and \(\alpha,\beta\colon L\to X\) are homotopic after composing with
    \(\eta_X\colon X\to\cosk_n X\),
    \ie~\(\eta_X\circ\alpha\simeq\eta_X\circ\beta\), then \(\alpha\simeq\beta\).
  \end{enumerate}
\end{lemma}
\begin{proof}
  Suppose \(X\) is some simplicial set, \(K\) is \(n\)--skeletal and \(L\) is
  \((n-1)\)--skeletal for some \(n\).
  \begin{enumerate}
  \item Since being \(n\)--skeletal is preserved under colimits and every
    simplicial set is a colimit of standard simplices, we need only consider the
    case \(K = \Delta^i\). \viktor{triangulate \(\Delta^i\times\Delta^1\)}
  \item Because \(K\) is assumed to be \(n\)--skeletal, the counit
    \(\varepsilon\colon \sk_nK \to K\) is an isomorphism. It follows that we
    have an isomorphism
    \[
      \psi\colon \hom(K,X)\to[\varepsilon^{*}]\hom(\sk_nK,
      X)\to\hom(K,\cosk_nX).
    \]
    It remains to check that \(\psi = (\eta_X)_{*}\). Take \(\gamma\colon K\to
    X\). Then, writing \(\iota = \iota_n\), we find as in
    \autoref{rem:unit-is-unit} and using the triangle identity
    \(\iota^{*}(\counit^{\iota_{!}\dashv\iota^{*}}_K) \circ
    \unit^{\iota_{!}\dashv\iota^{*}}_{\iota^{*}(K)} = {\id}\), that
    \begin{align*}
      \psi(\gamma) &=
                     \iota_{*}\Big(\iota^{*}(\gamma\circ\varepsilon)\circ\unit^{\iota_{!}\dashv\iota^{*}}_{\iota^{*}(K)}\Big)\circ
                     \eta_{K} =\\
                   &=
                     \cosk_n(\gamma)\circ\iota_{*}(\iota^{*}\counit^{\iota_{!}\dashv\iota^{*}}_K)\circ\iota_{*}(\unit^{\iota_{!}\dashv\iota^{*}}_{\iota^{*}(K)})\circ\eta_{K}
                     =\\
                   &= \cosk_n(\gamma)\circ\eta_K =\\
                   &= \eta_{X}\circ\gamma.
    \end{align*}
    The last equation follows from the naturality of \(\eta\).
  \item Since homotopy of maps \(L\to X\) is determined by maps
    \(L\times\Delta^1\to X\), this follows directly by combining (i) and
    (ii).\qedhere
  \end{enumerate}
  \viktor{write proof}
\end{proof}


\begin{proposition}\label{prop:cosk-unit-is-connected}
  The unit map \(\eta\colon X\to \cosk_n X\) is an \((n-1)\)--equivalence.
\end{proposition}
\begin{proof}
  By \autoref{prop:cosk-is-truncated} we can assume without loss of generality
  that both \(X\) and \(\cosk_n X\) are Kan complexes. For each \(i\leq n-1\)
  and \(x\in X_{0} = (\cosk_n X)_0\) we will construct an inverse map
  \(\varphi\colon \pi_{i}(\cosk_n X, x)\to \pi_i(X,x)\) to \(\eta_{*}\). Because
  of \(\pi_{n}(\cosk_n X, x) = 0\) and \autoref{lem:connected-homotopy-grps},
  this will be enough to establish the proposition.

  By \autoref{lem:skeletal-simp-sets}, the unit map \(\eta\colon X\to\cosk_n X\)
  induces an isomorphism
  \[
    \eta_{*}^{-1}\colon\hom(\sk_i\Delta^{i+1},\cosk_n
    X)\to[{\isom}]\hom(\sk_i\Delta^{i+1}, X)
  \]
  which preserves pointed maps. Composing with the map
  \(\hom_{*}(\sk_i\Delta^{i+1},X)\to\pi_i(X,x)\) which takes pointed homotopy
  classes, we obtain a map
  \[
    \overline\varphi\colon\hom_{*}(\sk_i\Delta^{i+1},\cosk_nX)\to\pi_{i}(X,x).
  \]
  Again, by \autoref{lem:skeletal-simp-sets} this map is compatible with pointed
  homotopy and therefore descends to a map \(\varphi\colon \pi_i(\cosk_nX,
  x)\to\pi_{i}(X,x)\).

  We compute:
  \begin{align*}
    \varphi(\eta_{*}([\alpha])) &= \varphi([\eta\circ\alpha]) =
                                  [\eta_{*}^{-1}(\eta\circ\alpha)] =
                                  [\alpha] \\
    \shortintertext{and}
    \eta_{*}(\varphi([\beta])) &= \eta_{*}([\eta_{*}^{-1}(\beta)]) =
                                 [\eta\circ\eta_{*}^{-1}(\beta)] = [\beta].\qedhere
  \end{align*}
\end{proof}

\begin{corollary}\label{cor:cosk-is-truncation}
  The functor \(\cosk_n\) together with the unit transformation \(\id\to
  \cosk_{n}\) is an \Th{(n-1)} truncation functor.\qed
\end{corollary}

% \begin{definition}\label{defn:rel-trunc-functor}
%   Let \(f\colon X\to Y\) be some map of simplicial sets and let
%   \((\trunc{n},\eta)\) be an \Th{n} truncation functor in the sense of
%   \autoref{defn:trunc_functor}. The associated \emph{relative \Th{n}
%   truncation} of \(f\) is defined as follows. Write \(Y\) as a homotopy
%   colimit of contractible simplicial sets,~e.\,g.
%   \[
%     Y \simeq \hocolim_{i\in I} \{i\}
%   \]
%   and observe that \(f\) is weakly equivalent to the map
%   \[
%     \hocolim_{i\in I} \hofib_i(f) \to \hocolim_{i\in I} \{i\}.
%   \]
%   Then we set
%   \[
%     \trunc{n}(f) = \trunc{n}^{Y}(X) = \hocolim_{i\in I} \trunc{n}\hofib_i(f).
%   \]
% \end{definition}

% % \begin{example}
% %   Using \autoref{cor:cosk-is-truncation}, our usual example of a
% %   relative truncation in the sense of \autoref{defn:rel-trunc-functor}
% %   will be the \emph{relative coskeleton}
% %   \[
% %     \cosk_n^Y(X) = Y\times^h_{\cosk_n Y}\cosk_nX.
% %   \]
% % \end{example}

% We can formally eliminate the dependence of homotopy groups on a chosen base
% point by taking a relative point of view. Any simplicial set \(X\) acquires a
% \emph{canonical} base point when pulled back to itself. That is, the diagonal
% \(X\to X\times X\) should be regarded as a base point of \(X\times X\)
% considered as a space parameterized by \(X\). As usual we can now consider
% homotopy classes of maps from spheres to \(X\times X\) over \(X\).

% \begin{definition}\label{defn:homotopy-sheaf}
%   Let \(X\) be any simplicial set. Let \(S^{n}\) be some simplicial model of
%   the \(n\)--sphere and consider the function complex \(X^{S^n}\). It
%   naturally is a space over \(X\) via the map \(\ev\colon X^{S^n}\to X\)
%   induced by evaluation at the canonical base point of \(S^n\). Let
%   \(\trunc{0}^X\) be a relative \Th{0} truncation functor. Then the
%   \emph{\Th{n} homotopy sheaf} of \(X\) is defined as
%   \[
%     \pi_nX = \trunc{0}^{X}(X^{S^n}).
%   \]
%   This is a priori a simplicial set equipped with a map \(\pi_n X\to X\) whose
%   homotopy fibers are \(0\)-truncated.
% \end{definition}

% \begin{proposition}
%   Let \(X\) be some simplicial set and \(x\in X_{0}\). Then the homotopy fiber
%   of \(\pi_nX\) over \(x\) is isomorphic to the usual homotopy group
%   \(\pi_n(X,x)\).
% \end{proposition}
% \begin{proof}
%   \viktor{write new proof}
% \end{proof}

% \begin{proposition}
%   Given any map \(f\colon X\to Y\), we have a long exact sequence of homotopy
%   sheaves:
% \end{proposition}

% \viktor{Prove classical homotopy group statements: relative truncation
% functors; long exact sequence; base point free definition}

\end{document}

% Local Variables:
% tex-main-file: "main.tex"
% End:

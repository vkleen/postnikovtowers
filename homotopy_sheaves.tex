\documentclass[main.tex]{subfiles}
\begin{document}
\section{Homotopy Sheaves}

Given a space \(X\) in classical homotopy theory one can always choose a base
point \(x\in X\) and define homotopy groups \(\pi_{n}(X,x) = [S^{n}, (X,x)]_{*}
= \pi_{0}\RHom_{*}(S^{n}, (X,x))\). For \(n\geq 1\) the group structure comes
from the usual (homotopy) cogroup structure on \(S^{n}\): the multiplication is
induced by precomposition with the \enquote{pinch map} \(S^{n}\to S^{n}\vee
S^{n}\).

The main difficulty in working with homotopy groups in more general homotopy
theories is that for various purposes restricting one's attention to pointed
spaces is not good enough. This phenomenon is entirely analogous to the
observation that a sheaf of sets on a topological space need not have any global
section. For example, the sheaf of local sections of a torsor on a topological
space has a global section if and only if the torsor is trivial. One could now
imagine working with a more general notion of base point for objects in an
arbitrary model category; for example, for simplicial sheaves there certainly
exist sections over small enough open sets. Doing so amounts to choosing a
presentation of a combinatorial model category in the sense of \cite{mr1870516}.

In light of these considerations we will attempt to treat all possible base
points at the same time. Let us first spell out the case of simplicial sets. Let
\(X\) be some simplicial set (connected or not). Choose some simplicial model
\(S^{n}\) for the \(n\)--sphere. Then we can consider the simplicial set
\(X^{S^{n}}\) of all simplicial maps \(S^{n}\to X\). Evaluation at the canonical
base point of \(S^{n}\) induces a map \(p\colon X^{S^{n}}\to X\). Observe that
the homotopy fiber of \(p\) over some point \(x\in X\) is precisely
\(\RHom_{*}(S^{n}, (X,x))\), the space of \emph{pointed} maps from \(S^{n}\) to
\((X,x)\). So the set of connected components of this homotopy fiber is
precisely \(\pi_{n}(X,x)\). In this sense, the \(0\)--truncation of the
canonical map \(p\colon X^{S^{n}}\to X\) captures the \(n\textsuperscript{th}\)
homotopy groups of \(X\) at all possible base points. This motivates the
definition of homotopy sheaves in any combinatorial simplicial model category.

\begin{definition}
  Let \(\mathbf M\) be a combinatorial simplicial model category and let
  \(X\in\mathbf M\). Then the \emph{\Th{n} homotopy sheaf} of \(X\) is the
  homotopy type \(\pi_{n}(X)\coloneqq\trunc{0}^{X}(X^{S^{n}})\in\disc(\mathbf
  M/X)\).
\end{definition}

The sheaves \(\pi_{n}(X)\) admit canonical sections. Indeed, the unique map
\(S^{n}\to *\) induces a map \(X\simeq X^{*}\to X^{S^{n}}\) which is a section
of the map \(X^{S^{n}}\to X\). Applying \(\trunc{0}^{X}\) yields a section of
\(\pi_{n}(X)\). Hence, the homotopy sheaves of \(X\) can be considered as
pointed objects in \(\disc(\mathbf M/X)\). However, to construct a group
structure on higher homotopy sheaves we will need to assume some exactness
properties for \(\mathbf M\). For instance, assuming that \(\mathbf M\) is a
model topos in the sense of \cite{rezkhomotopytoposes} would suffice.

\begin{definition}\label{defn:model-topos}
  A combinatorial simplicial model category \(\mathbf M\) is a \emph{model
    topos} if there is a presentation of \(\mathbf M\) as a \emph{left exact}
  Bousfield localization of a category of simplicial presheaves. That is,
  \(\mathbf M\) is a model topos if there is a simplicial Quillen
  adjunction\footnote{Written in the direction of the right adjoint.}
  \[
    \begin{mytikzcd}
      \mathbf M \arrow[r, shift right=.7ex, "i"'] & \arrow[l, shift right=.7ex,
      "L"'] \spshv(\mathbf C)
    \end{mytikzcd}
  \]
  for some simplicial category \(\mathbf C\) and the injective model structure
  on \(\spshv(\mathbf C)\) such that \(\LL L\) preserves finite homotopy limits
  and \(\RR i\) is \emph{homotopically fully faithful}, \ie~it induces weak
  equivalences
  \[
    \RHom(X, Y)\simeq\RHom(\RR i(X), \RR i(Y))
  \]
  for all \(X,Y\in\mathbf M\).
\end{definition}

Rezk proves in \cite[Corollary~6.10]{rezkhomotopytoposes} that for \(X\) an
object in a model topos \(\mathbf M\), the slice category \(\mathbf M/X\) is
again a model topos.

% However, for applications to motivic homotopy theory, the setting of model
% toposes is inconvenient. The \(\AA^1\)--model structure on simplicial presheaves
% on smooth schemes over some base is not a model topos. For this reason we will
% need to work in a more general setting than model toposes.\viktor{prove
%   \(\AA^1\)--homotopy is not a model topos}

% \begin{definition}[{cf.~\cite[Section~7]{gepner-kock}}]
%   A \emph{locally cartesian closed category \(C\)} is a category with finite
%   limits such that for every morphism \(f\colon X\to Y\) in \(C\), the
%   associated pullback functor \(f^{*}\colon C/Y \to C/X\) has a right adjoint
%   \(f_{*}\).

%   A \emph{locally cartesian closed model category \(\mathbf M\)} is a locally
%   cartesian closed category equipped with a model structure such that the
%   adjunctions \(f^{*}\dashv f_{*}\) for \(f\) a fibration between fibrant
%   objects are Quillen adjunctions for the slice model structures.
% \end{definition}

% \begin{definition}[{cf.~\cite{cisinski2002theories} and
%     \cite{cisinski2006prefaisceaux}}]
%   A \emph{Cisinski model structure} on a Grothendieck topos \(\mathbf M\) is a
%   cofibrantly generated model structure whose cofibrations are precisely the
%   monomorphisms.
% \end{definition}

% \begin{proposition}[{cf.~\cite{gepner-kock} and \cite{cisinski2002theories}}]\label{prop:proper-cisinski}
%   A proper Cisinski model category \(\mathbf M\) is locally cartesian closed.
% \end{proposition}
% \begin{proof}
%   Since \(\mathbf M\) is assumed to be a Grothendieck topos, it is automatically
%   locally cartesian closed. Hence, it remains to show that, for \(f\colon A\to
%   B\) a fibration between fibrant objects in \(\mathbf M\), the pullback functor
%   \(f^*\) is a left Quillen functor. By assumption it preserves cofibrations
%   because those are exactly the monomorphisms in a Cisinski model structure.
%   Consequently, we need to check that the pullback of an acyclic cofibration
%   \(g\colon X\to B\) along \(f\) is a weak equivalence. But this is immediate by
%   right properness.
% \end{proof}

% \begin{corollary}
%   The Morel--Voevodsky \(\AA^1\)--model structure on \(\spshv(\Sm{S})\) for a
%   base scheme \(S\) is locally cartesian closed, as are all its slices.
% \end{corollary}
% \begin{proof}
%   The \(\AA^1\)--model structure is constructed in~\cite{mv} as a left Bousfield
%   localization of the Nisnevich--local injective model structure on
%   \(\spshv(\Sm{S})\). As such, its underlying category is a Grothendieck topos
%   and its cofibrations are precisely the monomorphisms. Furthermore,~\cite{mv}
%   proves that it is proper. Hence, \autoref{prop:proper-cisinski} implies that
%   it is a locally cartesian closed model category, as are all its slices.
% \end{proof}

\begin{lemma}\label{lem:trunc-presheaves}
  Let \(\mathbf C\) be a simplicial category and let \(F\in\spshv(\mathbf C)\)
  be a simplicial presheaf. Then, for every \(c\in\mathbf C\), there is a natural
  weak equivalence \((\trunc{n}F)(c)\simeq \trunc{n}F(c)\) of simplicial sets.
\end{lemma}
\begin{proof}
  This follows from the fact that homotopy limits in \(\spshv(\mathbf C)\) may
  be computed pointwise and \autoref{prop:left-exact-preserves-trunc}.
\end{proof}

\begin{proposition}[{see \cite[Lemma~6.5.1.2]{mr2522659}}]\label{prop:trunc-products}
  If \(\mathbf M\) is a model topos, then the truncation functors \(\trunc{n}\)
  preserve finite homotopy products.
\end{proposition}
\begin{proof}
  First, let \(F,G\in\spshv(\mathbf C)\) be simplicial presheaves on some
  simplicial category \(\mathbf C\) and consider the canonical morphism
  \[
    \trunc{n}(F\times G)\to \trunc{n}F\times\trunc{n}G.
  \]
  To prove that it is a weak equivalence, it will be enough to check this after
  evaluating at all \(c\in\mathbf C\). Hence, by \autoref{lem:trunc-presheaves}
  we have reduced to the case of simplicial sets. There it follows from
  Whitehead's theorem by computing homotopy groups of both sides.

  Now choose a simplicial Quillen adjunction
  \[
    \begin{mytikzcd}
      \mathbf M \arrow[r, shift right=.7ex, "i"'] & \arrow[l, shift right=.7ex,
      "L"'] \spshv(\mathbf C)
    \end{mytikzcd}
  \]
  as in \autoref{defn:model-topos} and take \(X, Y\in\mathbf M\). By our
  previous considerations we have a natural weak equivalence \(\trunc{n}(\RR
  i(X)\times^{h}\RR i(Y))\simeq \trunc{n}\RR i(X)\times^{h}\trunc{n}\RR i(Y)\).
  Since \(\RR i\) is homotopically fully faithful the counit of the adjunction
  \(\LL L\dashv \RR i\) induces a weak equivalence \(\LL L\,\RR i (Z)\simeq Z\)
  for all \(Z\). Consequently, \autoref{prop:left-exact-preserves-trunc}
  provides us with weak equivalences
  \begin{align*}
    \trunc{n}(X\times^{h} Y) &\simeq \trunc{n}(\LL L\,\RR i(X\times^{h} Y)) \simeq \LL L\trunc{n}(\RR i(X)\times^{h} \RR i(Y)) \simeq \\
                             &\simeq \LL L(\trunc{n}\RR i(X)\times^{h} \trunc{n}\RR i(Y)) \simeq \\
                             &\simeq \trunc{n}(\LL L\,\RR i(X)) \times^{h} \trunc{n}(\LL L\,\RR i(Y)) \simeq \\
                             &\simeq \trunc{n}(X) \times^{h} \trunc{n}(Y)
  \end{align*}
  since \(\LL L\) is assumed to preserve finite homotopy limits.
\end{proof}

\Autoref{prop:trunc-products} allows us to produce a group structure on
\(\pi_{n}(X)\) for \(n\geq 1\) if \(X\) is an object in a model topos \(\mathbf
M\): We have the usual (homotopy) cogroup structure on \(S^{n}\) given by the
pinch map \(S^{n}\to S^{n}\vee S^{n}\). This induces a morphism
\[
  X^{S^{n}}\times^{h}_{X} X^{S^{n}} \simeq X^{S^{n}}\times^{h}_{X^{*}} X^{S^{n}}
  \simeq X^{S^{n}\vee S^{n}} \to X^{S^{n}}.
\]
By \autoref{prop:trunc-products} the functor \(\trunc{0}^{X}\) preserves finite
homotopy products in \(\mathbf M/X\), so we get a morphism
\[
  \pi_{n}(X)\times_{X}\pi_{n}(X)\isom
  \trunc{0}^{X}(X^{S^{n}})\times_{X}\trunc{0}^{X}(X^{S^{n}})\isom
  \trunc{0}^{X}(X^{S^{n}\vee S^{n}}) \to \trunc{0}^{X}(X^{S^{n}}) \isom
  \pi_{n}(X)
\]
in \(\disc(\mathbf M/X)\). It is then straightforward to check that for \(n\geq
1\) this gives group objects \(\pi_{n}(X)\) in \(\disc(\mathbf M/X)\), which are
abelian for \(n\geq 2\).

Given a map \(f\colon X\to Y\) in \(\mathbf M\) we can consider it as an object
of \(\mathbf M/Y\). As such, \(f\) has homotopy sheaves
\(\pi_{n}(f)\in\disc((\mathbf M/Y)/f)\). There is an equivalence of categories
\((\mathbf M/Y)/f\simeq \mathbf M/X\) which turns out to be a simplicial Quillen
equivalence. Hence, we can consider the homotopy sheaves \(\pi_{n}(f)\) as
objects of \(\disc(\mathbf M/X)\).

One can give a slightly more concrete description of \(\pi_{n}(f)\) as follows.
Consider the homotopy fiber product \(X^{S^{n}}\times^{h}_{Y^{S^{n}}} Y\) along
the canonical section \(Y\to Y^{S^{n}}\). This is a model for the object
\(f^{S^{n}}\in \mathbf M/Y\). Hence, along the equivalence \((\mathbf
M/Y)/f\simeq \mathbf M/X\), we find that
\[
  \trunc{0}^{X}(X^{S^{n}}\times^{h}_{Y^{S^{n}}} Y)\to X
\]
is a model for the homotopy sheaves \(\pi_{n}(f)\in\disc(\mathbf M/X)\).

\begin{remark}\label{rem:homotopy-group-functoriality}
  Rezk proves in \cite[Example~6.13]{rezkhomotopytoposes} that any morphism
  \(f\colon X\to Y\) in a model topos \(\mathbf M\) induces a simplicial Quillen
  adjunction
  \[
    \begin{mytikzcd}
      \mathbf M/X \arrow[r, shift right=.7ex, "f_{*}"'] & \arrow[l, shift
      right=.7ex, "f^{*}"'] \mathbf M/Y
    \end{mytikzcd}
  \]
  where the left adjoint \(f^{*}\) is given by homotopy pullback along \(f\) and
  is left exact. Consequently, \autoref{prop:left-exact-preserves-trunc} shows
  that \(f^{*}\) commutes with truncation. There is a homotopy commutative
  diagram
  \[
    \begin{mytikzcd}
      X^{S^{n}} \arrow[r] \arrow[d] & f^{*}Y^{S^{n}} \arrow[r] \arrow[d] & Y^{S^{n}} \arrow[d] \\
      X \arrow[r, equals] & X \arrow[r, "f"'] & Y
    \end{mytikzcd}
  \]
  and applying \(\trunc{0}^{X}\), we obtain a morphism
  \[
    f_{*}\colon \pi_{n}(X)\simeq\trunc{0}^{X}(X^{S^{n}})\longrightarrow
    \trunc{0}^{X}(f^{*}Y^{S^{n}})\simeq f^{*}\trunc{0}^{Y}(Y^{S^{n}})\simeq
    f^{*}\pi_{n}(Y).
  \]
\end{remark}

\begin{proposition}[{see~\cite[Remark~6.5.1.5]{mr2522659}}]\label{prop:long-exact-sequence}
  Given maps \(f\colon X\to Y\) and \(g\colon Y\to Z\) in a model topos
  \(\mathbf M\), there is a long exact sequence
  \[
    \cdots \to f^{*}\pi_{n+1}(g) \to \pi_{n}(f) \to \pi_{n}(g\circ f) \to
    f^{*}\pi_{n}(g) \to \cdots
  \]
  of pointed objects in \(\disc(\mathbf M/X)\).
\end{proposition}
\begin{proof}
  We will first construct the relevant maps between homotopy sheaves. First,
  \(g\) induces a natural map
  \[
    g_{*}\colon X^{S^{n}}\times^{h}_{Y^{S^{n}}} Y\to
    X^{S^{n}}\times^{h}_{Z^{S^{n}}} Z
  \]
  and by functoriality of truncation we get at map \(g_{*}\colon
  \pi_{n}(f)\to\pi_{n}(g\circ f)\). Second, applying
  \autoref{rem:homotopy-group-functoriality} to the model topos \(\mathbf M/Z\)
  yields the required morphism \(f_{*}\colon \pi_{n}(g\circ f)\to
  f^{*}\pi_{n}(g)\). \viktor{finish constructing the connecting homomorphism and
    prove exactness}
\end{proof}

\begin{proposition}\label{prop:truncated-homotopy-groups}
  If \(X\) is \(n\)--truncated, then \(\pi_iX = *\) for all \(i>n\). Conversely,
  if \(X\) is \(n\)--truncated and \(\pi_iX = *\) for all \(i>k\) for some
  \(k\), then \(X\) is \(k\)--truncated as well.
\end{proposition}
\begin{proof}
  \viktor{write proof.}
\end{proof}

\begin{proposition}\label{prop:connected-homotopy-groups}
  Let \(n\geq -1\). An object \(X\) in a model topos is \(n\)--connected if and
  only if it is \((-1)\)--connected and \(\pi_iX = *\) for all \(i\leq n\).
\end{proposition}
\begin{proof}
  \viktor{write proof.}
\end{proof}

\begin{proposition}[{cf.~\cite[Prop.~6.2.3.4]{mr2522659} and
    \cite[Prop.~7.8]{rezkhomotopytoposes}}]\label{prop:cech-complex}
  Let \(X\) be some object in a model topos. Let \(U_\bullet\) be the simplicial
  object with \(U_n = X^{n+1}\). Then \(X = U_0\to\hocolim U_\bullet\) is a
  \((-1)\)--truncation of \(X\).
\end{proposition}
\begin{proof}
  \viktor{write proof}
\end{proof}

\begin{proposition}{\viktor{add reference}}
  In a model topos, if \(g\colon Z\to X\) is \((-1)\)--connected and \(f\colon
  Y\to X\) is any morphism, then \(f\) is an equivalence if and only if the
  homotopy pullback \(g^*(f)\colon Y\times_X^h Z\to Z\) is an equivalence.
\end{proposition}
\begin{proof}
  \viktor{write proof}
\end{proof}
\end{document}

% Local Variables:
% tex-main-file: "main.tex"
% End:
